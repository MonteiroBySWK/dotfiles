Neste trabalho, foram apresentados os resultados parciais do desenvolvimento do sistema para a empresa júnior. Durante este período, foram realizadas as tarefas previstas no cronograma. As atividades iniciais de revisão sobre o back-end e avaliação das tecnologias ocorreram de setembro a novembro de 2024, enquanto a implementação do sistema e a realização de testes se estenderam de outubro de 2024 a janeiro de 2025. A simulação do sistema e a avaliação dos resultados têm sido realizadas de dezembro de 2024 até o presente momento. Os testes preliminares indicam que o sistema está funcionando de forma estável, dentro dos recursos disponíveis no plano gratuito da Railway. O desempenho tem sido satisfatório, com tempos de resposta adequados e consumo de recursos dentro dos limites estabelecidos. No entanto, otimizações futuras podem incluir a implementação de técnicas de cache e ajustes nas consultas ao banco de dados para melhorar a eficiência e reduzir a latência do sistema. As próximas etapas do projeto envolvem a ampliação da cobertura de testes, a documentação completa do sistema e o desenvolvimento de uma API RESTful, com o objetivo de viabilizar futuras integrações e, principalmente, possibilitar a aplicação multiplataforma e ampliar a escalabilidade do sistema. O resultado obtido até o momento foi a criação de um sistema funcional e eficiente para o gerenciamento de dados administrativos, financeiros e de clientes, com base nas tecnologias do framework web Django e PostgreSQL.