A Empresa Júnior do curso de Bacharelado em Engenharia de Computação, do Centro de Ciências Tecnológicas (CCT) da Universidade Estadual do Maranhão (UEMA), denominada Technos, tem como objetivo prestar serviço à comunidade interna e externa à universidade, com o treinamento em produtos tecnológicos, conserto e manutenção de equipamentos eletrônicos, com excelência de qualidade e baixo custo, procurando introduzir os participantes na realidade diária de uma empresa prestadora de serviços em hardware e Software.

A Technos reiniciou suas operações em 2022 com o atendimento aos alunos, professores e laboratório do CCT e de outros centros da UEMA, com os serviços de conserto e manutenção de computadores e equipamentos eletrônicos. Para o desenvolvimento dos trabalhos executados pela Technos, é necessária maior organização e visualização, tanto nas redes sociais quanto nos procedimentos administrativos e técnicos, por meio da construção de um sistema informatizado para registro das atividades, participantes, produtos e para a criação da identidade visual da empresa júnior.

Uma empresa júnior de Engenharia de Computação de acordo com a Lei 13.267 de 06 de abril de 2016, é uma “entidade organizada nos termos desta Lei, sob a forma de associação civil gerida por estudantes matriculados em cursos de graduação de instituições de ensino superior, com o propósito de realizar projetos e serviços que contribuam para o desenvolvimento acadêmico e profissional dos associados, capacitando-os para o mercado de trabalho” (BRASIL, 2016).

Para o desenvolvimento dos trabalhos da Technos e a profissionalização das atividades realizadas, é necessário um sistema de gestão de informação e controle dos produtos, serviços, materiais, apresentação da empresa e demais atividades, de forma segura e vinculada às melhores práticas de planejamento e execução do mercado. Assim, um sistema multiplataforma é um passo importante no caminho para o crescimento da empresa júnior e de seus participantes. O back-end dos sistemas computacionais são responsáveis pelo funcionamento do sistema coleta das amostras, com o tratamento das variáveis e a alocação dos dados no banco de dados, além da garantia de integridade dos dados armazenados (SILVA JÚNIOR, et al., 2023).

A proposta deste trabalho é o desenvolvimento de um back-end para um sistema multiplataforma para a empresa júnior do curso de Bacharelado em Engenharia de Computação da UEMA, a Technos.