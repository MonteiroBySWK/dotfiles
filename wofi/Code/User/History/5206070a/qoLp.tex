O sistema desenvolvido adota uma arquitetura monolítica, utilizando o framework Django e o banco de dados relacional PostgreSQL. Essa escolha foi motivada pela necessidade de um desenvolvimento ágil e pela possibilidade de reutilizar componentes robustos e seguros já consolidados na estrutura do Django, o que acelerou a implementação das funcionalidades essenciais.
A interface do sistema foi construída com o Django Admin, customizado para atender às demandas específicas da empresa júnior. Isso permitiu que a equipe tivesse acesso imediato às funcionalidades, eliminando a necessidade de desenvolver uma interface gráfica do zero e reduzindo o tempo de adoção da plataforma.
Para a implementação do sistema, foram desenvolvidos três módulos principais:
Gerenciamento de Clientes: Responsável pelo armazenamento e gerenciamento de informações dos clientes atendidos pela empresa júnior;
Serviços: Módulo que controla os pedidos realizados, permitindo o acompanhamento do status (pendente, em andamento e concluído);
Finanças: Responsável pelo registro das movimentações financeiras, incluindo entradas e saídas de capital, além da atualização automática do saldo do caixa com base nos pedidos concluídos.
Cada um desses módulos foi implementado como um aplicativo separado dentro do Django, garantindo organização e modularidade. Para garantir a estabilidade do sistema, foram desenvolvidos testes unitários para cada um dos modelos para possíveis falhas e garantir maior confiabilidade.
O sistema foi implantado para testes na plataforma Railway, utilizando um plano gratuito que disponibiliza $US 5,00$ em recursos. Para a implantação, foi empregada a conteinerização com Docker, juntamente com a configuração do Gunicorn para o gerenciamento eficiente das requisições. Durante esse processo, foram necessários ajustes nos parâmetros de configuração do Django e do PostgreSQL para garantir o funcionamento adequado em ambiente de produção.
Nos primeiros testes, o sistema apresentou um tempo médio de resposta de aproximadamente 2,8 segundos, com um consumo máximo de 200 MB de RAM, 0,3 vCPU e 100 MB de armazenamento. Esses valores indicam um funcionamento estável dentro dos limites do plano disponível.
Atualmente, os membros da empresa júnior já utilizam o sistema para gerenciar clientes, pedidos e movimentações financeiras. A frequência de uso varia conforme a demanda por serviços e a necessidade de atualização dos dados administrativos.