\documentclass[12pt, a4paper]{article}
\usepackage[utf8]{inputenc} % Codificação de entrada
\usepackage[T1]{fontenc}    % Codificação de fonte
\usepackage[brazilian]{babel} % Idioma português do Brasil
\usepackage{amsmath}        % Para equações matemáticas
\usepackage{graphicx}       % Para incluir imagens
\usepackage{float}          % Para controlar a posição de figuras e tabelas
\usepackage[hidelinks]{hyperref} % Para links clicáveis no PDF (hidelinks remove as caixas coloridas)
\usepackage[numbers]{natbib} % Para citações e bibliografia (se não usar abntex2)
\usepackage{indentfirst}    % Para indentar o primeiro parágrafo de cada seção
\usepackage{caption}        % Para legendas de figuras e tabelas
\usepackage{subcaption}     % Para subfiguras
\usepackage{geometry}       % Para configurar margens
\geometry{a4paper, top=3cm, bottom=2cm, left=3cm, right=2cm} % Margens ABNT


\title{Desenvolvimento de um Back-end para Sistema Multiplataforma da Technos}
\author{Gabriel Silva Monteiro}
\date{\today}

\begin{document}
\maketitle

\tableofcontents

\section*{Resumo}
\input{secoes/resumo.tex}
\section*{Palavras-chave}
\input{secoes/palavras_chave.tex}
\section{Introdução}

\input{secoes/introducao.tex}
\section{Objetivos}
\input{secoes/objetivos.tex}
\section{Metodologia}
\input{secoes/metodologia.tex}
\section{Resultados}
\input{secoes/resultados.tex}
\section{Conclusão}
\input{secoes/conclusao.tex}
\section*{Referências}
\bibliographystyle{plain}
\bibliography{referencias}

\end{document}