% ---
% pacotes
% ---
\documentclass[
	%options
	% -- opções de estilo ---
	article, % tipo de trabalho acadêmico (ver abntex2-modelo-*.tex)
	%chapter, % para cada capítulo em arquivos separados
	oneside,   % para impressão em apenas um lado do papel (versão para web)
	% -- opções de layout ---
	12pt,    % tamanho da fonte
	% --- opções de idioma ---
	portuguese, % para o idioma portugues
	]{abntex2}

% ---
% Pacotes básicos
% ---
\usepackage{lmodern}           % Usa a fonte Latin Modern
\usepackage[T1]{fontenc}       % Selecao de codigos de fonte.
\usepackage[utf8]{inputenc}    % Codificacao do documento (conversao com o LaTeX).
\usepackage{lastpage}          % Usado pela Ficha catalográfica
\usepackage{indentfirst}       % Indenta o primeiro parágrafo de cada seção.
\usepackage{color}             % Controle das cores.
\usepackage{graphicx}          % Inclusão de gráficos
\usepackage{microtype}         % para melhorias de justificação
\usepackage{subfig}            % para subfiguras (se for usar)
\usepackage{float}             % para controlar o posicionamento de figuras e tabelas
\usepackage{amsmath}           % para símbolos e estruturas matemáticas
\usepackage{amssymb}           % para símbolos matemáticos adicionais
\usepackage{multirow}          % para usar multirow em tabelas
\usepackage{array}             % para configurações de coluna em tabelas
\usepackage{booktabs}          % para linhas de tabela com melhor aparência
\usepackage{ragged2e}          % Para controle de quebra de linha e alinhamento
\usepackage[hidelinks]{hyperref}          % para criar links clicáveis
\usepackage{url}               % para formatar URLs

% ---
% Pacotes de citações
% ---
\usepackage[
	abbreviate=true,
	backend=bibtex,
	style=abnt-numeric, % ou abnt-alf para autoria-data
	]{abntex2cite}

% ---
% Configurações do pacote 'graphicspath'
% ---
\graphicspath{{images/}} % Certifique-se de que suas imagens estão nesta pasta

% ---
% Informações do trabalho (para capa, folha de rosto, etc.)
% ---
\titulo{Desenvolvimento de um Back-End para um Sistema Multiplataforma para a Empresa Júnior do Curso de Bacharelado em Engenharia de Computação da UEMA}
\autor{Gabriel Silva Monteiro}
\local{São Luís - MA}
\data{2025} % Mês e Ano de defesa/publicação, ex: \data{Junho de 2025}
\orientador{Prof. Pedro Brandão Neto}
\instituicao{
  Universidade Estadual do Maranhão -- UEMA \\
  Centro de Ciências Tecnológicas -- CCT \\
  Departamento de Engenharia de Computação
}
% --- IMPORTANTE: AJUSTE ESTAS LINHAS PARA RELATÓRIO FINAL ---
\tipotrabalho{Relatório Final de Iniciação Científica}
\preambulo{Relatório final apresentado ao Programa de Iniciação Científica (PIBIC) da Universidade Estadual do Maranhão, como parte dos requisitos para a conclusão do projeto.}

% ---
% Dados para a ficha catalográfica (opcional, mas comum em relatórios finais)
% ---
\makeatletter
\def\ABNTEX@ficha@catalograca{%
  \null\vfil
  \begin{center}
  \bfseries FICHA CATALOGRÁFICA
  \end{center}
  \vspace{0.5cm}
  \begin{center}
  \begin{minipage}{10cm}
  \small
  \textit{Dados Internacionais de Catalogação na Publicação (CIP)}

  (Câmara Brasileira do Livro, SP, Brasil)

  \begin{tabular}{ll}
  & Monteiro, Gabriel Silva \\
  & Desenvolvimento de um Back-End para um Sistema Multiplataforma para a Empresa Júnior do Curso de Bacharelado em Engenharia de Computação da UEMA / Gabriel Silva Monteiro. -- São Luís, MA, 2025. \\
  & 41 f. : il. \\
  & Orientador: Prof. Pedro Brandão Neto. \\
  & \textbf{Relatório Final de Iniciação Científica} -- Universidade Estadual do Maranhão, Centro de Ciências Tecnológicas, Departamento de Engenharia de Computação, São Luís, 2025. \\ % ATUALIZADO AQUI
  & 1. Desenvolvimento de software. 2. Back-end. 3. Django. 4. PostgreSQL. I. Neto, Pedro Brandão. II. Universidade Estadual do Maranhão. Centro de Ciências Tecnológicas. Departamento de Engenharia de Computação. III. Título. \\
  & \hspace{5.5cm} CDD (23. ed.) 005.1 \\
  \end{tabular}
  \textit{Elaborado por [Nome do Bibliotecário], CRB [Número do CRB]}
  \end{minipage}
  \end{center}
  \vfil\null
}
\makeatother

% ---
% Arquivo com as referências bibliográficas
% ---
\bibliography{referencias.bib}

% ---
% Início do documento
% ---
\begin{document}

% Elementos pré-textuais
\imprimircapa           % Imprime a capa
\imprimirfolhaderosto    % Imprime a folha de rosto

% --- FOLHA DE APROVAÇÃO (DESCOMENTE E PREENCHA AO FINAL) ---
% A ABNT exige para trabalhos finais. Preencha as datas e nomes dos avaliadores.
% \begin{folhadeaprovacao}
%   \banca{
%     \membro{Prof. Dr. [Nome do Orientador]}{Orientador}{UEMA}
%     \membro{Prof. Dr. [Nome do Avaliador 1]}{Avaliador Externo}{Instituição do Avaliador 1}
%     \membro{Prof. Dr. [Nome do Avaliador 2]}{Avaliador Interno}{UEMA}
%   }
%   \dataaprovacao{21}{junho}{2025} % Dia, mês por extenso, ano
% \end{folhadeaprovacao}

% Resumo e palavras-chave
\begin{resumo}
  O objetivo deste trabalho é o desenvolvimento de um back-end para um sistema multiplataforma destinado à Technos, empresa júnior do curso de Bacharelado em Engenharia de Computação do Centro de Ciências Tecnológicas da Universidade Estadual do Maranhão. A Technos busca apoiar a formação dos alunos proporcionando o contato dos alunos com a prestação de serviços de hardware e software à comunidade interna e externa à universidade, com o treinamento em produtos tecnológicos, conserto e manutenção de equipamentos eletrônicos, com excelência de qualidade e busca contínua de inovação. A implementação do back-end será realizada utilizando o framework web Django e o sistema de gerenciamento de banco de dados PostgreSQL. O sistema permitirá à Technos gerenciar dados administrativos, financeiros e de clientes de forma eficiente e segura, contribuindo para a organização e crescimento da empresa. O projeto visa também proporcionar aos alunos uma experiência prática no desenvolvimento de sistemas reais, aplicando os conhecimentos adquiridos em sala de aula.
\end{resumo}

\begin{palavraschave}
  Desenvolvimento de Software, Back-End, Django, PostgreSQL.
\end{palavraschave}

% Listas (abntex2 já inclui os comandos)
\listoffigures         % Lista de figuras
% \listoftables          % Lista de tabelas (comente se não tiver tabelas)

\sumario               % Gera o sumário (abntex2 tem seu próprio sumário)

% Conteúdo textual
\textual               % Comando do abntex2 para iniciar a numeração das páginas

\section{Introdução}
A Technos, empresa júnior do curso de Bacharelado em Engenharia de Computação da Universidade Estadual do Maranhão (UEMA), é uma iniciativa que visa complementar a formação acadêmica dos alunos, oferecendo-lhes a oportunidade de aplicar conhecimentos teóricos em projetos práticos. Criada com o propósito de aproximar os estudantes do mercado de trabalho, a Technos atua na prestação de serviços de hardware e software para a comunidade interna e externa à universidade, promovendo o desenvolvimento de habilidades técnicas e gerenciais \cite{BRASIL:2016}.

Desde sua fundação, a empresa tem se dedicado a oferecer soluções tecnológicas, incluindo conserto e manutenção de equipamentos eletrônicos, desenvolvimento de sistemas e treinamento em produtos tecnológicos. A Technos reiniciou suas operações em 2022 com o atendimento aos alunos, professores e laboratório do CCT e de outros centros da UEMA, com os serviços de conserto e manutenção de computadores e equipamentos eletrônicos. Com o crescimento das atividades, surgiu a necessidade de um sistema robusto e eficiente para gerenciar as demandas administrativas, financeiras e de clientes.

A ausência de um sistema integrado tem gerado desafios na organização e controle das informações, impactando a eficiência dos processos internos da empresa júnior. Atualmente, o controle é feito de forma manual ou por meio de planilhas, o que dificulta o acompanhamento do fluxo de caixa, o registro de serviços prestados e o gerenciamento de dados dos clientes. Essa lacuna operacional justifica a importância do desenvolvimento de um back-end para um sistema multiplataforma, que será capaz de centralizar e automatizar essas operações.

O objetivo principal deste trabalho é desenvolver a arquitetura de back-end para um sistema multiplataforma que atenda às necessidades específicas da Technos. Este sistema permitirá o gerenciamento de dados administrativos, financeiros e de clientes, garantindo a integridade e a segurança das informações. A escolha do framework web Django \cite{DJANGO:2025} e do sistema de gerenciamento de banco de dados PostgreSQL \cite{THEPOSTGRESQLGLOBALDEVELOPMENTGROUP:2025} fundamenta-se na robustez, escalabilidade e na vasta comunidade de desenvolvedores, que oferece suporte e recursos para a construção de aplicações complexas e de alta performance. Além disso, o projeto visa proporcionar aos alunos envolvidos uma experiência prática valiosa no desenvolvimento de sistemas reais, alinhada às melhores práticas da engenharia de software, contribuindo para a sua formação profissional \cite{SILVAJUNIOR:2023}.


\section{Desenvolvimento}
O projeto consiste no desenvolvimento de um sistema multiplataforma para a empresa júnior Technos, com foco no back-end. A estrutura do projeto é dividida em módulos, conforme as funcionalidades necessárias para o gerenciamento da empresa.

\subsection{Tecnologias Utilizadas}
\subsubsection{Django}
Django é um framework web de alto nível em Python que incentiva o desenvolvimento rápido e um design limpo e pragmático. Construído por desenvolvedores experientes, ele cuida de grande parte do trabalho de desenvolvimento web, para que o desenvolvedor possa se concentrar na escrita do aplicativo sem a necessidade de reinventar a roda. É gratuito e de código aberto \cite{DJANGO:2025}.

\subsubsection{PostgreSQL}
PostgreSQL é um poderoso sistema de gerenciamento de banco de dados objeto-relacional de código aberto com mais de 30 anos de desenvolvimento ativo e uma forte reputação de confiabilidade, robustez de recursos e desempenho. É altamente extensível, suportando muitos tipos de dados SQL e JSON, e possui otimização de consulta avançada \cite{THEPOSTGRESQLGLOBALDEVELOPMENTGROUP:2025}.

\subsection{Módulos do Sistema}
O sistema será composto pelos seguintes módulos principais:
\begin{itemize}
    \item \textbf{Autenticação e Autorização:} Gerenciamento de usuários, perfis de acesso e permissões.
    \item \textbf{Clientes:} Cadastro e consulta de informações de clientes.
    \item \textbf{Serviços:} Registro e acompanhamento dos serviços prestados.
    \item \textbf{Financeiro:} Controle de receitas e despesas.
    \item \textbf{Relatórios:} Geração de relatórios gerenciais.
\end{itemize}

\section{Resultados e Discussão}
Até o momento, o desenvolvimento do back-end para o sistema multiplataforma da Technos progrediu conforme o cronograma. A estrutura inicial do projeto foi definida, e os primeiros módulos estão em fase de implementação e testes.

\subsection{Estrutura do Projeto}
A estrutura do back-end foi organizada em aplicações Django, cada uma responsável por um conjunto específico de funcionalidades. Essa modularização facilita a manutenção e a escalabilidade do sistema. As principais aplicações são:
\begin{itemize}
    \item `accounts`: Responsável pela autenticação e autorização de usuários.
    \item `clients`: Gerencia o cadastro e as informações dos clientes.
    \item `services`: Controla o registro e o status dos serviços.
    \item `financial`: Lida com as transações financeiras.
\end{itemize}

\subsection{Desenvolvimento dos Módulos}
O módulo de autenticação e autorização foi implementado utilizando os recursos nativos do Django, o que agiliza o processo e garante a segurança. As funcionalidades de registro de usuários e login estão operacionais. O sistema foi implantado para testes na plataforma Railway, utilizando um plano gratuito que disponibiliza US\$5,00 em recursos \cite{Griffiths:1987tj}. A integração com o PostgreSQL também foi realizada, e o banco de dados está configurado para armazenar as informações dos usuários e suas permissões.

A interface do sistema foi construída com o Django Admin, customizado para atender às demandas específicas da empresa júnior. A personalização do Django Admin permite que a Technos gerencie dados de forma intuitiva, mesmo sem a necessidade de desenvolver uma interface front-end complexa neste momento. Isso acelera a entrega de valor e permite que a empresa comece a utilizar o sistema para gerenciamento básico.

\subsection{Desafios e Próximos Passos}
Um dos principais desafios tem sido a modelagem do banco de dados para garantir a flexibilidade necessária para futuras expansões do sistema, principalmente no que diz respeito às particularidades de cada tipo de serviço oferecido pela Technos. A segurança dos dados também é uma prioridade, e estão sendo implementadas boas práticas de segurança, como validação de dados e proteção contra ataques comuns da web.

Os próximos passos incluem a finalização dos módulos de clientes, serviços e financeiro, a implementação de testes automatizados para garantir a qualidade do código e a criação de APIs RESTful para integração com futuras aplicações front-end (web e mobile). Será feita uma refatoração do código existente para garantir a conformidade com padrões de código e a otimização de desempenho.

\section{Considerações Finais}
Neste trabalho, foram apresentados os resultados do desenvolvimento do back-end para o sistema multiplataforma da Technos. Durante este período, foram realizadas as tarefas previstas no cronograma. As atividades iniciais de revisão sobre o back-end e avaliação das tecnologias ocorreram de setembro a novembro de 2024, enquanto a implementação do sistema e a realização de testes se estenderam de outubro de 2024 a janeiro de 2025. A simulação do sistema e a avaliação dos resultados têm sido realizadas de dezembro de 2024 até o presente momento. Os testes preliminares indicam que o sistema está funcionando de forma estável, dentro dos recursos disponíveis no plano gratuito da Railway. O desempenho tem sido satisfatório, com tempos de resposta adequados e consumo de recursos dentro dos limites estabelecidos. No entanto, otimizações futuras podem incluir a implementação de técnicas de cache e ajustes nas consultas ao banco de dados para melhorar a eficiência e reduzir a latência do sistema. As próximas etapas do projeto envolvem a ampliação da cobertura de testes, a documentação completa do sistema e o desenvolvimento de uma API RESTful, com o objetivo de viabilizar futuras integrações e, principalmente, possibilitar a aplicação multiplataforma e ampliar a escalabilidade do sistema. O resultado obtido até o momento foi a criação de um sistema funcional e eficiente para o gerenciamento de dados administrativos, financeiros e de clientes, com base nas tecnologias do framework web Django e PostgreSQL.

\begin{figure}[H]
    \centering
    \includegraphics[width=0.8\textwidth]{exemplo_diagrama_arquitetura.png}
    \caption{Exemplo de Diagrama de Arquitetura do Sistema}
    \label{fig:diagrama_arquitetura}
\end{figure}

\end{textual}

% ---
% Referências
% ---
\bibliographystyle{abntex2-numeric} % Ou abntex2-alf para estilo autor-data
\bibliography{referencias.bib}

\end{document}