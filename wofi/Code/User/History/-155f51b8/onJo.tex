\documentclass[12pt, a4paper]{article}
\usepackage[utf8]{inputenc} % Codificação de entrada
\usepackage[T1]{fontenc}    % Codificação de fonte
\usepackage[brazilian]{babel} % Idioma português do Brasil
\usepackage{amsmath}        % Para equações matemáticas
\usepackage{graphicx}       % Para incluir imagens
\usepackage{float}          % Para controlar a posição de figuras e tabelas
\usepackage[hidelinks]{hyperref} % Para links clicáveis no PDF (hidelinks remove as caixas coloridas)
\usepackage[numbers]{natbib} % Para citações e bibliografia (se não usar abntex2)
\usepackage{indentfirst}    % Para indentar o primeiro parágrafo de cada seção
\usepackage{caption}        % Para legendas de figuras e tabelas
\usepackage{subcaption}     % Para subfiguras
\usepackage{geometry}       % Para configurar margens
\geometry{a4paper, top=3cm, bottom=2cm, left=3cm, right=2cm} % Margens ABNT


\title{Desenvolvimento de um Back-End para um Sistema Multiplataforma para a Empresa Júnior do Curso de Bacharelado em Engenharia de Computação da UEMA}
\author{Gabriel Silva Monteiro \\ Orientador: Prof. Pedro Brandão Neto}
\date{São Luís - MA \\ 2025}

\begin{document}
\maketitle

\tableofcontents

\begin{abstract}
    \noindent O objetivo deste trabalho é o desenvolvimento de um back-end para um sistema multiplataforma destinado à Technos, empresa júnior do curso de Bacharelado em Engenharia de Computação do Centro de Ciências Tecnológicas da Universidade Estadual do Maranhão. A Technos busca apoiar a formação dos alunos proporcionando o contato dos alunos com a prestação de serviços de hardware e software à comunidade interna e externa à universidade, com o treinamento em produtos tecnológicos, conserto e manutenção de equipamentos eletrônicos, com excelência de qualidade e baixo custo. Para a melhoria e organização da empresa é proposto um sistema multiplataforma para organização das demandas, registro dos materiais permanentes e de uso contínuo, cadastros de usuários e clientes, com a integração de vendas de serviços e apresentação da empresa. O trabalho deve pesquisar as melhores tecnologias de banco de dados, conexão interna do sistema, segurança da informação, escalabilidade e versatilidade para um sistema confiável, procurando garantir a confidencialidade, integridade e disponibilidade dos serviços a serem prestados.
    \vspace{0.5cm} % Adiciona um pequeno espaço
    \noindent \textbf{Palavras-chave:} Empresa júnior, sistema multiplataforma, back-end, serviços.
\end{abstract}


\input{secoes/palavras_chave.tex}
\section{Introdução}

\input{secoes/introducao.tex}
\section{Objetivos}
\input{secoes/objetivos.tex}
\section{Metodologia}
\input{secoes/metodologia.tex}
\section{Resultados}
\input{secoes/resultados.tex}
\section{Conclusão}
\input{secoes/conclusao.tex}
\section*{Referências}
\bibliographystyle{plain}
\bibliography{referencias}

\end{document}