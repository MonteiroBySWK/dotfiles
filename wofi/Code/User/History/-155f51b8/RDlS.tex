\documentclass[
    12pt,                 % Tamanho da fonte
    a4paper,              % Tamanho do papel
    sumario=tradicional,  % Estilo do sumário (tradicional ou expandido)
    chapter=TITLE,        % Capítulos como títulos numerados (mais comum para relatórios)
    section=TITLE,        % Seções como títulos numerados (dentro dos "capítulos")
    subsection=TITLE,     % Subseções como títulos numerados
]{abntex2}

% --- PACOTES ---
% Pacotes para formatação ABNT
\usepackage{abntex2cite}     % Estilos de citação e bibliografia ABNT

% Pacotes essenciais
\usepackage[utf8]{inputenc}  % Codificação de entrada
\usepackage[T1]{fontenc}     % Codificação de fonte
\usepackage{lmodern}         % Fonte Latin Modern (melhora a aparência)
\usepackage[brazilian]{babel}% Idioma português do Brasil
\usepackage{indentfirst}     % Indenta o primeiro parágrafo de cada seção

% Pacotes para gráficos e tabelas
\usepackage{graphicx}        % Para incluir imagens
\usepackage{float}           % Para controlar a posição de figuras e tabelas
\usepackage{caption}         % Para legendas de figuras e tabelas
\usepackage{subcaption}      % Para subfiguras

% Pacotes para matemática
\usepackage{amsmath, amsfonts, amssymb} % Pacotes matemáticos

% Pacotes para links
\usepackage[hidelinks]{hyperref} % Para links clicáveis no PDF (hidelinks remove as caixas coloridas)


% --- INFORMAÇÕES DO DOCUMENTO (PARA CAPA E FOLHA DE ROSTO) ---
\autor{Gabriel Silva Monteiro}
\titulo{Desenvolvimento de um Back-End para um Sistema Multiplataforma para a Empresa Júnior do Curso de Bacharelado em Engenharia de Computação da UEMA}
\local{São Luís - MA}
\data{2025} % A data atual é 14 de junho de 2025. Se quiser a data automática, use \today
\orientador{Prof. Pedro Brandão Neto}
\instituicao{
  Universidade Estadual do Maranhão -- UEMA \\
  Centro de Ciências Tecnológicas -- CCT \\
  Departamento de Engenharia de Computação
}
\tipotrabalho{Relatório de Iniciação Científica}
\preambulo{Relatório parcial referente às atividades realizadas conforme o cronograma previsto no plano de trabalho da iniciação científica.}


% --- INÍCIO DO DOCUMENTO ---
\begin{document}

% Elementos pré-textuais (gerados pelo abntex2)
\imprimircapa           % Imprime a capa
\imprimirfolhaderosto    % Imprime a folha de rosto

% Resumo e Palavras-chave
\resumo{O objetivo deste trabalho é o desenvolvimento de um back-end para um sistema multiplataforma destinado à Technos, empresa júnior do curso de Bacharelado em Engenharia de Computação do Centro de Ciências Tecnológicas da Universidade Estadual do Maranhão. A Technos busca apoiar a formação dos alunos proporcionando o contato dos alunos com a prestação de serviços de hardware e software à comunidade interna e externa à universidade, com o treinamento em produtos tecnológicos, conserto e manutenção de equipamentos eletrônicos, com excelência de qualidade e baixo custo. Para a melhoria e organização da empresa é proposto um sistema multiplataforma para organização das demandas, registro dos materiais permanentes e de uso contínuo, cadastros de usuários e clientes, com a integração de vendas de serviços e apresentação da empresa. O trabalho deve pesquisar as melhores tecnologias de banco de dados, conexão interna do sistema, segurança da informação, escalabilidade e versatilidade para um sistema confiável, procurando garantir a confidencialidade, integridade e disponibilidade dos serviços a serem prestados.}

\palavraschave{Empresa júnior, sistema multiplataforma, back-end, serviços}

% Listas (abntex2 já inclui os comandos)
\listoffigures         % Lista de figuras
% \listoftables          % Lista de tabelas (comente se não tiver tabelas)

\sumario               % Gera o sumário (abntex2 tem seu próprio sumário)

% --- ELEMENTOS TEXTUAIS ---
\textual               % Comando do abntex2 para iniciar a numeração das páginas

\section{Introdução}
A Empresa Júnior do curso de Bacharelado em Engenharia de Computação, do Centro de Ciências Tecnológicas (CCT) da Universidade Estadual do Maranhão (UEMA), denominada Technos, tem como objetivo prestar serviço à comunidade interna e externa à universidade, com o treinamento em produtos tecnológicos, conserto e manutenção de equipamentos eletrônicos, com excelência de qualidade e baixo custo, procurando introduzir os participantes na realidade diária de uma empresa prestadora de serviços em hardware e Software. A Technos reiniciou suas operações em 2022 com o atendimento aos alunos, professores e laboratório do CCT e de outros centros da UEMA, com os serviços de conserto e manutenção de computadores e equipamentos eletrônicos. Para o desenvolvimento dos trabalhos executados pela Technos, é necessária maior organização e visualização, tanto nas redes sociais quanto nos procedimentos administrativos e técnicos, por meio da construção de um sistema informatizado para registro das atividades, participantes, produtos e para a criação da identidade visual da empresa júnior. Uma empresa júnior de Engenharia de Computação de acordo com a Lei 13.267 de 06 de abril de 2016, é uma “entidade organizada nos termos desta Lei, sob a forma de associação civil gerida por estudantes matriculados em cursos de graduação de instituições de ensino superior, com o propósito de realizar projetos e serviços que contribuam para o desenvolvimento acadêmico e profissional dos associados, capacitando-os para o mercado de trabalho” \cite{BRASIL2016}. Para o desenvolvimento dos trabalhos da Technos e a profissionalização das atividades realizadas, é necessário um sistema de gestão de informação e controle dos produtos, serviços, materiais, apresentação da empresa e demais atividades, de forma segura e vinculada às melhores práticas de planejamento e execução do mercado. Assim, um sistema multiplataforma é um passo importante no caminho para o crescimento da empresa júnior e de seus participantes. O back-end dos sistemas computacionais são responsáveis pelo funcionamento do sistema coleta das amostras, com o tratamento das variáveis e a alocação dos dados no banco de dados, além da garantia de integridade dos dados armazenados \cite{SILVAJUNIOR2023}. A proposta deste trabalho é o desenvolvimento de um back-end para um sistema multiplataforma para a empresa júnior do curso de Bacharelado em Engenharia de Computação da UEMA, a Technos.

\section{Objetivos}
\subsection{Objetivos Gerais}
O objetivo geral do trabalho é desenvolver um back-end para um sistema multiplataforma para a empresa júnior do curso de Bacharelado em Engenharia de Computação da UEMA.

\subsection{Objetivos Específicos}
\begin{itemize}
    \item Pesquisar sobre as principais tecnologias utilizadas na construção de um back-end para um sistema multiplataforma de acordo com os requisitos da empresa júnior;
    \item Analisar as melhores ferramentas para uso no desenvolvimento do back-end do sistema multiplataforma da empresa júnior;
    \item Implementar as melhores soluções de acordo com as necessidades e requisitos da empresa júnior;
    \item Avaliar o protótipo com a integração do front-end desenvolvido, por meio de teste de software e do uso em um sistema real;
    \item Produzir um capítulo de livro a partir dos trabalhos realizados para submissão e possível publicação.
\end{itemize}

\section{Metodologia}
A pesquisa está dividida em três etapas:
\begin{enumerate}
    \item Levantamento bibliográfico sobre as tecnologias de back-end para uso no desenvolvimento do sistema multiplataforma;
    \item Desenvolvimentos de testes em um sistema protótipo, com a implementação e testes em tempo real;
    \item Avaliação dos resultados obtidos com a adequação do sistema às necessidades e requisitos de sistema da Technos.
\end{enumerate}
Na primeira etapa será realizada o levantamento do estado da arte sobre as tecnologias de back-end que podem ser utilizadas no desenvolvimento do projeto. Na segunda parte devem ser realizadas a implementação das melhores soluções obtidas, e na terceira parte os testes de adequações finais do programa com a implantação em um sistema real, e validação pela empresa júnior. A partir dos resultados obtidos deve ser preparado e submetido para publicação um capítulo de livro.

\section{Resultados} % Mantendo "Resultados" como seção principal
O sistema desenvolvido adota uma arquitetura monolítica, utilizando o framework Django e o banco de dados relacional PostgreSQL. Essa escolha foi motivada pela necessidade de um desenvolvimento ágil e pela possibilidade de reutilizar componentes robustos e seguros já consolidados na estrutura do Django, o que acelerou a implementação das funcionalidades essenciais. A interface do sistema foi construída com o Django Admin, customizado para atender às demandas específicas da empresa júnior. Isso permitiu que a equipe tivesse acesso imediato às funcionalidades, eliminando a necessidade de desenvolver uma interface gráfica do zero e reduzindo o tempo de adoção da plataforma.

% As figuras devem estar na mesma pasta do arquivo .tex
\begin{figure}[h!]
    \centering
    \includegraphics[width=0.9\textwidth]{figura1_dashboard.png} % Substitua por seu arquivo de imagem
    \caption{Visão geral dos aplicativos no dashboard do sistema da Technos.}
    \label{fig:dashboard}
    \small Fonte: O Autor, 2025.
\end{figure}

Para a implementação do sistema, foram desenvolvidos três módulos principais:
\begin{itemize}
    \item \textbf{Gerenciamento de Clientes:} Responsável pelo armazenamento e gerenciamento de informações dos clientes atendidos pela empresa júnior;
    \item \textbf{Serviços:} Módulo que controla os pedidos realizados, permitindo o acompanhamento do status (pendente, em andamento e concluído);
\end{itemize}

\begin{figure}[h!]
    \centering
    \includegraphics[width=0.9\textwidth]{figura2_pedidos.png} % Substitua por seu arquivo de imagem
    \caption{Interface do sistema, demonstrando a tela de Pedidos, de serviços, com detalhes de cada solicitação.}
    \label{fig:pedidos}
    \small Fonte: O Autor, 2025.
\end{figure}

\begin{itemize}
    \item \textbf{Finanças:} Responsável pelo registro das movimentações financeiras, incluindo entradas e saídas de capital, além da atualização automática do saldo do caixa com base nos pedidos concluídos.
\end{itemize}

\begin{figure}[h!]
    \centering
    \includegraphics[width=0.9\textwidth]{figura3_financas.png} % Substitua por seu arquivo de imagem
    \caption{Interface do sistema com informações financeiras e de acesso.}
    \label{fig:financas}
    \small Fonte: O Autor, 2025.
\end{figure}

Cada um desses módulos foi implementado como um aplicativo separado dentro do Django, garantindo organização e modularidade. Para garantir a estabilidade do sistema, foram desenvolvidos testes unitários para cada um dos modelos para possíveis falhas e garantir maior confiabilidade. O sistema foi implantado para testes na plataforma Railway, utilizando um plano gratuito que disponibiliza US\$5,00 em recursos. Para a implantação, foi empregada a conteinerização com Docker, juntamente com a configuração do Gunicorn para o gerenciamento eficiente das requisições. Durante esse processo, foram necessários ajustes nos parâmetros de configuração do Django e do PostgreSQL para garantir o funcionamento adequado em ambiente de produção. Nos primeiros testes, o sistema apresentou um tempo médio de resposta de aproximadamente 2,8 segundos, com um consumo máximo de 200 MB de RAM, 0,3 vCPU e 100 MB de armazenamento. Esses valores indicam um funcionamento estável dentro dos limites do plano disponível. Atualmente, os membros da empresa júnior já utilizam o sistema para gerenciar clientes, pedidos e movimentações financeiras. A frequência de uso varia conforme a demanda por serviços e a necessidade de atualização dos dados administrativos.

\section{Considerações Parciais}
Neste trabalho, foram apresentados os resultados parciais do desenvolvimento do sistema para a empresa júnior. Durante este período, foram realizadas as tarefas previstas no cronograma. As atividades iniciais de revisão sobre o back-end e avaliação das tecnologias ocorreram de setembro a novembro de 2024, enquanto a implementação do sistema e a realização de testes se estenderam de outubro de 2024 a janeiro de 2025. A simulação do sistema e a avaliação dos resultados têm sido realizadas de dezembro de 2024 até o presente momento. Os testes preliminares indicam que o sistema está funcionando de forma estável, dentro dos recursos disponíveis no plano gratuito da Railway. O desempenho tem sido satisfatório, com tempos de resposta adequados e consumo de recursos dentro dos limites estabelecidos. No entanto, otimizações futuras podem incluir a implementação de técnicas de cache e ajustes nas consultas ao banco de dados para melhorar a eficiência e reduzir a latência do sistema. As próximas etapas do projeto envolvem a ampliação da cobertura de testes, a documentação completa do sistema e o desenvolvimento de uma API RESTful, com o objetivo de viabilizar futuras integrações e, principalmente, possibilitar a aplicação multiplataforma e ampliar a escalabilidade do sistema. O resultado obtido até o momento foi a criação de um sistema funcional e eficiente para o gerenciamento de dados administrativos, financeiros e de clientes, com base nas tecnologias do framework web Django e PostgreSQL.

% --- ELEMENTOS PÓS-TEXTUAIS ---
% O abntex2 gerencia as referências automaticamente com \bibliography
\bibliography{referencias} % Onde 'referencias.bib' é o nome do seu arquivo .bib

\end{document}