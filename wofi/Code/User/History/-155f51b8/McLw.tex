\documentclass[a4paper,12pt]{article}
\bibliographystyle{plain}





\begin{document}

%Remove numeração da página atual
\thispagestyle{empty}

% Como fazer um cabeçalho passo a passo com tabular: https://www.youtube.com/watch?v=NEvF9mJwOXI&list=PLF6ZF9NW0WmqUAgtkYlQmCDHP6H_bYwGk&index=3

\begin{tabular}[l]{ll}
\multirow{5}*{\includegraphics[width=50pt]{logo.png}} & 
\textbf{\resizebox{!}{0.3cm}{Fundação Universidade Federal do ABC}}\\&
\textbf{\resizebox{!}{0.3cm}{Pró reitoria de pesquisa}} \\
& \textbf{\resizebox{!}{0.25cm}{Av. dos Estados, 5001, Santa Terezinha, Santo André/SP, CEP 09210-580}}\\
& \textbf{\resizebox{!}{0.25cm}{Bloco L, 3ºAndar, Fone (11) 3356-7617}}\\
& \textbf{\resizebox{!}{0.25cm}{iniciacao@ufabc.edu.br}} \\
\end{tabular}
\\
\\
\\
\\
\\
\\
\\
\\
\\
\\
\\
\\
\\
\\
\\
\\
\\
Projeto de Iniciação Científica submetido para avaliação no Edital: XX/XXXX
\\
\\
\\
\\
\\
\\
\\
\\
\\
\\
\\
\\
\\
\\
\\
\\
\\
\\
\textbf{Título do projeto:} O projeto não deve conter nenhuma indicação quanto ao nome do orientador e do discente.
\\
\\
\textbf{Palavras-chave do projeto:} \\
\\
\textbf{Área do conhecimento do projeto:} \\

%Quebra de página
\newpage


%Sumário
\tableofcontents

\newpage

%Comando para determinar espaçamento entre linhas (1,5 nesse caso)
\onehalfspacing
{
%Define como tópico/seção do projeto (irá aparecer automaticamente no sumário)  
\section{Resumo}
Este é um modelo de projeto de Iniciação Científica elaborado pelo Comitê do Programa de Iniciação Científica (CPIC).

\section{Introdução}
 Este modelo foi criado com o intuito de facilitar a escrita do projeto de iniciação científica que deverá ser elaborado em conjunto entre o orientador e o discente. Em caso de dúvidas entrar em contato através do e-mail: iniciacao@ufabc.edu.br
Apesar de não haver um número mínimo e máximo de páginas, o comitê recomenda que o projeto tenha em torno de 5 a 10 páginas, sendo o mais objetivo, claro e sucinto possível.
O projeto não deve conter nenhuma indicação quanto ao nome do orientador e do discente. Possibilitando assim que o projeto seja avaliado da forma mais cega possível.
As citações e referências bibliográficas devem ser feitas, preferencialmente, no padrão ABNT. Exemplos:
Exemplo de citação de um livro (Sobrenome1 et al., Ano);
Exemplo de citação de um artigo de revista (Sobrenome1, Ano);
Exemplo de citação de um artigo de simpósio/congresso (Sobrenome1; Sobrenome2, Ano);
Na Seção de Introdução e Justificativa, orienta-se que se faça uma introdução contextualizando o trabalho, assim como uma breve justificativa para a realização do mesmo. \cite{Fluri_2019}


\section{Objetivos}
Os objetivos principais de um projeto de IC, independentemente da sua área de atuação, é a inserção do aluno no mundo da Pesquisa. De acordo com o CNPq, os principais objetivos do programa PIBIC são:


%Inicia seção de itens
\begin{itemize}
    \item despertar vocação científica e incentivar novos talentos entre estudantes de graduação;

    \item contribuir para reduzir o tempo médio de titulação de mestres e doutores;

    \item contribuir para a formação científica de recursos humanos que se dedicarão a qualquer atividade profissional;
    \item estimular uma maior articulação entre a graduação e pós-graduação;

    \item contribuir para a formação de recursos humanos para a pesquisa;

    \item contribuir para reduzir o tempo médio de permanência dos alunos na pós-graduação.

    \item estimular pesquisadores produtivos a envolverem alunos de graduação nas atividades científica, tecnológica e artístico-cultural;

    \item proporcionar ao bolsista, orientado por pesquisador qualificado, a aprendizagem de técnicas e métodos de pesquisa, bem como estimular o desenvolvimento do pensar cientificamente e da criatividade, decorrentes das condições criadas pelo confronto direto com os problemas de pesquisa; e

    \item ampliar o acesso e a integração do estudante à cultura científica.\\
    
Já o PIBITI, os principais objetivos são:\\

    \item Contribuir para a formação e inserção de estudantes em atividades de pesquisa, desenvolvimento tecnológico e inovação;

    \item Contribuir para a formação de recursos humanos que se dedicarão ao fortalecimento da capacidade inovadora das empresas no País, e
    \item Contribuir para a formação do cidadão pleno, com condições de participar de forma criativa e empreendedora na sua comunidade.\\
    
No Edital 01/2019, durante a submissão do projeto será solicitado que o orientador identifique se o projeto de pesquisa que será submetido se enquadra como PIBITI/CNPq. \cite{Cho:2017ufy}

\end{itemize}


\section{Metodologia}
Na Seção de Metodologia, orienta-se que se indique os principais métodos que serão utilizados para a realização da pesquisa, bem como os principais conceitos teóricos envolvidos.


\section{Viabilidade}
Na Seção de Viabilidade orienta-se que sejam descritos os materiais e/ou equipamentos que serão utilizados para a realização do trabalho, bem como se os mesmos se encontram disponíveis ou não.
Também poderá ser descrito se o projeto está vinculado a um projeto maior do orientador e, de acordo com a natureza do projeto, informar se haverá condições como espaço e equipamentos adequados à execução do mesmo, devendo sempre manter o anonimato dos autores.
Caso o projeto proposto necessite de aprovação da comissão de ética em pesquisa (CEP) e da comissão de ética no uso de animais (CEUA) solicita-se que o orientador busque a aprovação dstes órgãos com antecedência o bastante para que haja tempo suficiente para o discente cumprir todo o projeto proposto. Neste caso, solicita-se que este processo também seja indicado na Seção Cronograma de Atividades, indicando as datas previstas para submissão, análise e aprovação do mesmo. Mais informações podem ser encontradas em:\\ \cite{Griffiths:111880}

\begin{itemize}
    \item CEP - http://cep.ufabc.edu.br/index.php/en/
    \item CEUA - http://comissoes.ufabc.edu.br/ceua/
\end{itemize}

O comitê também entende que em alguns casos esta seção não se aplica, podendo optar por dizer na Seção de Viabilidade "Não se aplica".


\section{Cronograma de atividades}
\begin{itemize}
    \item Etapa 1
    \begin{itemize}
        \item Etapa 1.a.
        \item Etapa 1.b.
        \item Etapa 1.c
    \end{itemize}
        
    \item Etapa 2
    \begin{itemize}
        \item Etapa 2.a.
        \item Etapa 2.b.
        \item Etapa 2.c.
    \end{itemize}
    
    \item Etapa 3
    \begin{itemize}
        \item Etapa 3.a.
        \item Etapa 3.b.
        \item Etapa 3.c.
    \end{itemize}
\end{itemize}

% Como fazer matrizes e tabelas no Overleaf: https://www.youtube.com/watch?v=BHaeu70rCao

\begin{tabular}{|c|c|c|c|c|c|c|c|c|c|c|c|c|}\hline
    \multirow{2}*{Etapa} &
    \multicolumn{12}{c}{Mês}\\\cline{2-13}
      & 01 & 02 & 03 & 04 & 05 & 06 & 07 & 08 & 09 & 10 & 11 & 12 \\ \hline 
     1.a & x & x & x & & & & & & & & & \\ \hline
     1.b & & x & x & x & & & & & & & & \\ \hline
     1.c & & & x & x & x & & & & & & & \\ \hline
     2.a & & & & x & x & x & & & & & & \\ \hline
     2.b & & & & & x &x & x & & & & & \\ \hline
     2.c & & & & & & x & x & x & & & & \\ \hline
     3.a & & & & & & & x & x & x & & & \\ \hline
     3.b & & & & & & & & x & x & x & & \\ \hline
     3.c & & & & & & & & & x & x & x & x \\ \hline
\end{tabular}

}
\newpage

%Referências bibliográficas 

\bibliography{bibliografia.bib}


\end{document}

Sobre

    About us
    Our values
    Careers
    Press & awards
    Blog

Learn

    LaTeX in 30 minutes
    Modelos
    Webinars
    Tutorials
    How to insert images
    How to create tables

