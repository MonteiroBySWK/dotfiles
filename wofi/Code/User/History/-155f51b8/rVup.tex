\documentclass{article}
\usepackage[utf8]{inputenc}
\usepackage[brazil]{babel}

\author{Gabriel Silva Monteiro}
\date{\today}

\begin{document}

\section{Resumo}
O objetivo deste trabalho é o desenvolvimento de um back-end para um sistema multiplataforma destinado à Technos, empresa júnior do curso de Bacharelado em Engenharia de Computação do Centro de Ciências Tecnológicas da Universidade Estadual do Maranhão. A Technos busca apoiar a formação dos alunos proporcionando o contato dos alunos com a prestação de serviços de hardware e software à comunidade interna e externa à universidade, com o treinamento em produtos tecnológicos, conserto e manutenção de equipamentos eletrônicos, com excelência de qualidade e baixo custo. Para a melhoria e organização da empresa é proposto um sistema multiplataforma para organização das demandas, registro dos materiais permanentes e de uso contínuo, cadastros de usuários e clientes, com a integração de vendas de serviços e apresentação da empresa. O trabalho deve pesquisar as melhores tecnologias de banco de dados, conexão interna do sistema, segurança da informação, escalabilidade e versatilidade para um sistema confiável, procurando garantir a confidencialidade, integridade e disponibilidade dos serviços a serem prestados.

\section{Introdução}


\end{document}