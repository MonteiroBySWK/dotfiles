\documentclass[
    12pt,                 % Tamanho da fonte
    a4paper,              % Tamanho do papel
    sumario=tradicional,  % Estilo do sumário (tradicional ou expandido)
    chapter=TITLE,        % Capítulos como títulos numerados (mais comum para relatórios)
    section=TITLE,        % Seções como títulos numerados (dentro dos "capítulos")
    subsection=TITLE,     % Subseções como títulos numerados
    % Outras opções comuns da ABNT:
    % oldfontcommands,      % Usa comandos de fonte antigos (opcional, pode causar warnings)
    % subfignumselabel,     % Para subfiguras com numeração (ex: 1(a))
    % subfigbottomcap,      % Legenda de subfiguras na parte inferior
    % oneside,              % Para impressão em uma face do papel (padrão)
    % openright,            % Capítulo começa sempre na página direita (se oneside for falso)
    % twoside,              % Para impressão frente e verso
]{abntex2}

% --- PACOTES ---
% Pacotes para formatação ABNT
\usepackage{abntex2cite}     % Estilos de citação e bibliografia ABNT

% Pacotes essenciais
\usepackage[utf8]{inputenc}  % Codificação de entrada
\usepackage[T1]{fontenc}     % Codificação de fonte
\usepackage{lmodern}         % Fonte Latin Modern (melhora a aparência)
\usepackage{babel}   % Idioma português do Brasil
\usepackage{indentfirst}     % Indenta o primeiro parágrafo de cada seção

% Pacotes para gráficos e tabelas
\usepackage{graphicx}        % Para incluir imagens
\usepackage{float}           % Para controlar a posição de figuras e tabelas
\usepackage{caption}         % Para legendas de figuras e tabelas
\usepackage{subcaption}      % Para subfiguras

% Pacotes para matemática
\usepackage{amsmath, amsfonts, amssymb} % Pacotes matemáticos

% Pacotes para links
\usepackage[hidelinks]{hyperref} % Para links clicáveis no PDF (hidelinks remove as caixas coloridas)

% --- CONFIGURAÇÕES DE LAYOUT (ABNTEX2 já configura grande parte) ---
% Ajustes finos de margens se necessário (abntex2 já tem defaults bons)
% \usepackage[a4paper, top=3cm, bottom=2cm, left=3cm, right=2cm]{geometry}


% --- INFORMAÇÕES DO DOCUMENTO (PARA CAPA E FOLHA DE ROSTO) ---
\autor{Gabriel Silva Monteiro}
\titulo{Desenvolvimento de um Back-End para um Sistema Multiplataforma para a Empresa Júnior do Curso de Bacharelado em Engenharia de Computação da UEMA}
\local{São Luís - MA}
\data{\today} % Ou defina a data específica se necessário
\orientador{Prof. Pedro Brandão Neto}
\instituicao{
  Universidade Estadual do Maranhão -- UEMA \\
  Centro de Ciências Tecnológicas -- CCT \\
  Departamento de Engenharia de Computação
}
\tipotrabalho{Relatório de Iniciação Científica}
\preambulo{Relatório parcial referente às atividades realizadas conforme o cronograma previsto no plano de trabalho da iniciação científica.}


% --- INÍCIO DO DOCUMENTO ---
\begin{document}

% Elementos pré-textuais (gerados pelo abntex2)
\imprimircapa           % Imprime a capa
\imprimirfolhaderosto    % Imprime a folha de rosto

% Resumo e Palavras-chave
% O abntex2 já tem comandos específicos para isso
\resumo{O objetivo deste trabalho é o desenvolvimento de um back-end para um sistema multiplataforma destinado à Technos, empresa júnior do curso de Bacharelado em Engenharia de Computação do Centro de Ciências Tecnológicas da Universidade Estadual do Maranhão. A Technos busca apoiar a formação dos alunos proporcionando o contato dos alunos com a prestação de serviços de hardware e software à comunidade interna e externa à universidade, com o treinamento em produtos tecnológicos, conserto e manutenção de equipamentos eletrônicos, com excelência de qualidade e baixo custo. Para a melhoria e organização da empresa é proposto um sistema multiplataforma para organização das demandas, registro dos materiais permanentes e de uso contínuo, cadastros de usuários e clientes, com a integração de vendas de serviços e apresentação da empresa. O trabalho deve pesquisar as melhores tecnologias de banco de dados, conexão interna do sistema, segurança da informação, escalabilidade e versatilidade para um sistema confiável, procurando garantir a confidencialidade, integridade e disponibilidade dos serviços a serem prestados.}

\palavraschave{Empresa júnior, sistema multiplataforma, back-end, serviços}

% Listas (abntex2 já inclui os comandos)
\listoffigures         % Lista de figuras
% \listoftables          % Lista de tabelas (comente se não tiver tabelas)

\sumario               % Gera o sumário (abntex2 tem seu próprio sumário)

% --- ELEMENTOS TEXTUAIS ---
\textual               % Comando do abntex2 para iniciar a numeração das páginas

\section{Introdução}
\input{secoes/introducao.tex}

\section{Objetivos}
\input{secoes/objetivos.tex}

\section{Metodologia}
\input{secoes/metodologia.tex}

\section{Resultados} % Renomeei para "Resultados" conforme seu documento original, que usa "Resultados Parciais"
\input{secoes/resultados.tex}

\section{Considerações Parciais} % Mudei para "Considerações Parciais" conforme o texto do seu documento
\input{secoes/consideracoes_parciais.tex}

% --- ELEMENTOS PÓS-TEXTUAIS ---
% O abntex2 gerencia as referências automaticamente com \bibliography
\bibliography{referencias} % Onde 'referencias.bib' é o nome do seu arquivo .bib

\end{document}