\documentclass[openright]{normas-utf-tex} % openright = o capítulo começa sempre em páginas ímpares

% force A4 paper format
\special{papersize=210mm,297mm}

% Pacotes ABNT e de formatação
\usepackage[hidelinks,plainpages=true, bookmarks=true, bookmarksdepth=6, breaklinks, pdfstartview=FitH,pdftitle={Desenvolvimento de um Back-End para Sistema Multiplataforma},pdfauthor={Gabriel Silva Monteiro},pdfsubject={Engenharia de Computação},pdfkeywords={Empresa júnior; sistema multiplataforma; back-end; serviços; Django; PostgreSQL; Docker; Microsserviços; DevOps;}]{hyperref}
\usepackage[hyphenbreaks]{breakurl}
\urlstyle{same}

% Configuração correta das referências bibliográficas.
% Usando biblatex com style=abnt para a melhor conformidade e flexibilidade
% Note: O template original usa 'abntcite'. Se houver conflito ou preferência, pode-se ajustar.
% Para este exemplo, manterei o biblatex que usamos nas interações anteriores por ser mais robusto para a estrutura que construímos.
% Caso o seu ambiente LaTeX da UEMA seja configurado para 'abntcite', pode ser necessário uma adaptação.
\usepackage[backend=biber, style=abnt, natbib=true]{biblatex}
\addbibresource{references.bib} % Nome do arquivo .bib que você precisará criar

\usepackage[brazil]{babel} % pacote portugues brasileiro
\usepackage[utf8]{inputenc} % pacote para acentuacao direta

\usepackage{amsmath,amsfonts,amssymb} % pacote matematico
\usepackage{graphicx} % pacote grafico
\usepackage[T1]{fontenc}
\usepackage{lmodern}
%\usepackage[final]{pdfpages} % para inclusão de PDF externos, como a ata (opcional)
\usepackage{float} % Para controlar a posição das figuras (H = here)
\usepackage{pifont} % Para ticks e crosses (dingbats)
\newcommand{\cmark}{\ding{51}}%
\newcommand{\xmark}{\ding{55}}%

% ---------- Preambulo (Informações do Documento para normas-utf-tex) ----------
\instituicao{Universidade Estadual do Maranhão} % nome da instituicao
\instituicaoSigla{UEMA} % nome da instituicao
\programa{Programa Institucional de Bolsas de Iniciação Científica} % nome do programa
\programaSigla{PIBIC} % nome do programa
\area{Engenharia de Computação} % area de concentracao

\documento{Relatório de Iniciação Científica} % Tipo de documento
\documentoingles{Scientific Initiation Report}
\nivel{Graduação} % Nível do trabalho
\curso{Engenharia de Computação}
\titulacao{Bolsista de Iniciação Científica} % Título que o autor possui como bolsista

\titulo{Desenvolvimento de um Back-End para um Sistema Multiplataforma para a Empresa Júnior do Curso de Bacharelado em Engenharia de Computação da UEMA} % titulo do trabalho em portugues
\title{Development of a Back-End for a Cross-Platform System for the Junior Enterprise of the Bachelor's Degree in Computer Engineering at UEMA} % titulo do trabalho em ingles

\autor{Gabriel Silva Monteiro} % autor do trabalho
\cita{MONTEIRO, Gabriel} % sobrenome (maiusculas), nome do autor do trabalho (para citação na ficha catalográfica)

\palavraschave{Empresa júnior; sistema multiplataforma; back-end; serviços; Django; PostgreSQL; Docker; Microsserviços; DevOps;} % palavras-chave do trabalho em portugues
\keywords{Junior enterprise; cross-platform system; back-end; services; Django; PostgreSQL; Docker; Microservices; DevOps;} % palavras-chave do trabalho em ingles

% Comentário para a folha de rosto/ficha catalográfica.
% Adaptei para o contexto de um Relatório de Iniciação Científica.
\comentario{Relatório final de Iniciação Científica apresentado ao \UTFPRprogramadata\ (\UTFPRprogramaSigladata) da \ABNTinstituicaodata\ (\ABNTinstituicaoSigladata), como requisito parcial para a obtenção do título de Bacharel em Engenharia de Computação.}

\orientador{Prof. Pedro Brandão Neto} % nome do orientador do trabalho
%\coorientador{Nome do Coorientador} % nome do coorientador do trabalho, caso exista

\local{São Luís - MA} % cidade
\data{2025} % ano

% Desativa hifenizacao mantendo o texto justificado.
\tolerance=1
\emergencystretch=\maxdimen
\hyphenpenalty=10000
\hbadness=10000
\sloppy

%---------- Inicio do Documento ----------
\begin{document}
\pdfstringdefDisableCommands{%
	\let\MakeUppercase\relax
}

% Geracao automatica da capa e folha de rosto
\capa
\folhaderosto

% Inserção da ATA (se houver e for um PDF)
%\includepdf{ata.pdf}

% Dedicatoria (opcional)
\begin{dedicatoria}
    Aos meus pais, pelo apoio incondicional e incentivo constante.
    À minha família e amigos, pela paciência e compreensão durante esta jornada.
\end{dedicatoria}

% Agradecimentos (opcional)
\begin{agradecimentos}
    Agradeço primeiramente a Deus, por me guiar e conceder forças.
    Ao meu orientador, Prof. Pedro Brandão Neto, pela valiosa orientação, confiança e paciência, que foram fundamentais para a realização deste trabalho.
    À Universidade Estadual do Maranhão (UEMA) e ao Programa Institucional de Bolsas de Iniciação Científica (PIBIC), pelo suporte e oportunidade de desenvolver esta pesquisa.
    Aos membros da Technos, a Empresa Júnior de Engenharia de Computação, pela colaboração e pelo fornecimento dos requisitos que tornaram este projeto relevante e aplicável.
    Aos colegas e professores do Departamento de Engenharia de Computação, pelo ambiente de aprendizado e troca de conhecimentos.
\end{agradecimentos}

% Epígrafe (opcional)
\begin{epigrafe}
    "O único modo de fazer um excelente trabalho é amar o que você faz." \\ \vspace{0.5cm} \hspace*{\fill} -- Steve Jobs
\end{epigrafe}

% Resumo e Abstract
\begin{resumo}
    O presente trabalho aborda o desenvolvimento de um back-end robusto e escalável para um sistema multiplataforma, destinado à Technos, a empresa júnior de Engenharia de Computação da Universidade Estadual do Maranhão (UEMA). O propósito central foi otimizar a gestão interna da empresa, padronizando processos e centralizando informações. A fase inicial do projeto empregou \textbf{Python e o framework Django} para o desenvolvimento de um sistema monolítico, com \textbf{PostgreSQL} como sistema gerenciador de banco de dados relacional. Funcionalidades essenciais foram implementadas, incluindo módulos de autenticação de usuários, gestão de clientes, controle financeiro (registro fiscal e extrato), gerenciamento de demandas/atividades e controle de estoque de materiais. O sistema foi colocado em \textbf{produção em janeiro de 2025}, passando por uma transição de um ambiente de nuvem (Railway) para um \textbf{servidor local na UEMA}, utilizando \textbf{Docker} para conteinerização. Desafios técnicos inerentes à arquitetura monolítica do Django, como o alto acoplamento e dificuldades na depuração de erros decorrentes da tipagem dinâmica do Python, foram identificados e analisados. Tais observações fundamentam a justificativa para uma \textbf{futura migração para uma arquitetura de microsserviços baseada em Java e Spring}, visando aprimorar a manutenibilidade, escalabilidade e performance. Os resultados demonstram que o back-end em operação já contribui significativamente para a organização, acessibilidade de dados e automação de tarefas na Technos, estabelecendo uma base sólida para a evolução contínua da gestão da empresa.
\end{resumo}

\begin{abstract}
    The present work addresses the development of a robust and scalable back-end for a cross-platform system, intended for Technos, the junior enterprise of Computer Engineering at the State University of Maranhão (UEMA). The central purpose was to optimize the company's internal management, standardizing processes and centralizing information. The initial phase of the project employed \textbf{Python and the Django framework} for the development of a monolithic system, with \textbf{PostgreSQL} as the relational database management system. Essential functionalities were implemented, including user authentication modules, client management, financial control (fiscal registration and statement), demand/activity management, and material stock control. The system was put into \textbf{production in January 2025}, transitioning from a cloud environment (Railway) to a \textbf{local server at UEMA}, using \textbf{Docker} for containerization. Technical challenges inherent in Django's monolithic architecture, such as high coupling and difficulties in debugging errors arising from Python's dynamic typing, were identified and analyzed. These observations justify a \textbf{future migration to a microservices architecture based on Java and Spring}, aiming to improve maintainability, scalability, and performance. The results demonstrate that the operational back-end already contributes significantly to the organization, data accessibility, and task automation at Technos, establishing a solid foundation for the continuous evolution of the company's management.
\end{abstract}

% Listas (opcionais, mas recomenda-se a partir de 5 elementos)
\listadefiguras % geracao automatica da lista de figuras
%\listadetabelas % Descomente se tiver tabelas no seu documento
%\listadequadros % para quadros
%\listadesiglas % Para usar, defina as siglas com \sigla{ABRV}{Descricao Completa}
%\listadesimbolos % Para usar, defina os símbolos com \simbolo{Símbolo}{Descricao Completa}

% Sumário
\sumario % geracao automatica do sumario

%---------- Inicio do Texto ----------
% Aqui iniciamos o corpo principal do seu relatório, organizado em capítulos.

\chapter{Introdução}
\label{chap:introducao}

    A Empresa Júnior do curso de Bacharelado em Engenharia de Computação, do Centro de Ciências Tecnológicas (CCT) da Universidade Estadual do Maranhão (UEMA), denominada Technos, tem como objetivo prestar serviço à comunidade interna e externa à universidade, com o treinamento em produtos tecnológicos, conserto e manutenção de equipamentos eletrônicos, com excelência de qualidade e baixo custo, procurando introduzir os participantes na realidade diária de uma empresa prestadora de serviços em hardware e Software.

    A Technos reiniciou suas operações em 2022 com o atendimento aos alunos, professores e laboratório do CCT e de outros centros da UEMA, com os serviços de conserto e manutenção de computadores e equipamentos eletrônicos. Para o desenvolvimento dos trabalhos executados pela Technos, é necessária maior organização e visualização, tanto nas redes sociais quanto nos procedimentos administrativos e técnicos, por meio da construção de um sistema informatizado para registro das atividades, participantes, produtos e para a criação da identidade visual da empresa júnior.

    Uma empresa júnior de Engenharia de Computação de acordo com a Lei 13.267 de 06 de abril de 2016, é uma “entidade organizada nos termos desta Lei, sob a forma de associação civil gerida por estudantes matriculados em cursos de graduação de instituições de ensino superior, com o propósito de realizar projetos e serviços que contribuam para o desenvolvimento acadêmico e profissional dos associados, capacitando-os para o mercado de trabalho” \citep{BRASIL_2016}.

    Para o desenvolvimento dos trabalhos da Technos e a profissionalização das atividades realizadas, é necessário um sistema de gestão de informação e controle dos produtos, serviços, materiais, apresentação da empresa e demais atividades, de forma segura e vinculada às melhores práticas de planejamento e execução do mercado. Assim, um sistema multiplataforma é um passo importante no caminho para o crescimento da empresa júnior e de seus participantes. O back-end dos sistemas computacionais são responsáveis pelo funcionamento do sistema coleta das amostras, com o tratamento das variáveis e a alocação dos dados no banco de dados, além da garantia de integridade dos dados armazenados \citep{SILVA_JUNIOR_2023}.

    A proposta deste trabalho é o desenvolvimento de um back-end para um sistema multiplataforma para a empresa júnior do curso de Bacharelado em Engenharia de Computação da UEMA, a Technos.

\section{Objetivos}

\subsection{Objetivos Gerais}
    O objetivo geral do trabalho é desenvolver um back-end para um sistema multiplataforma para a empresa júnior do curso de Bacharelado em Engenharia de Computação da UEMA.

\subsection{Objetivos Específicos}
    \begin{itemize}[label=\textbullet]
        \item Pesquisar sobre as principais tecnologias utilizadas na construção de um back-end para um sistema multiplataforma de acordo com os requisitos da empresa júnior;
        \item Analisar as melhores ferramentas para uso no desenvolvimento do back-end do sistema multiplataforma da empresa júnior;
        \item Implementar as melhores soluções de acordo com as necessidades e requisitos da empresa júnior;
        \item Avaliar o protótipo com a integração do front-end desenvolvido, por meio de teste de software e do uso em um sistema real;
        \item Produzir um capítulo de livro a partir dos trabalhos realizados para submissão e possível publicação.
    \end{itemize}

\chapter{Metodologia}
\label{chap:metodologia}
    A pesquisa e o desenvolvimento do back-end foram conduzidos de forma iterativa, adaptando-se à natureza do projeto de Iniciação Científica e aos requisitos dinâmicos da empresa júnior. O processo foi estruturado em três etapas principais, que permearam o ciclo de vida do projeto:

    \section{Levantamento Bibliográfico e Análise de Tecnologias}
    Esta fase, ocorrida de setembro a novembro de 2024, envolveu uma pesquisa aprofundada sobre as principais tecnologias para o desenvolvimento de back-end em sistemas multiplataforma. O foco inicial recaiu sobre *frameworks* que promovessem agilidade no desenvolvimento e robustez para a aplicação de gestão. A pesquisa levou à escolha de \textbf{Python}, com o *framework* \textbf{Django} \citep{DJANGO_2025}, devido à sua reconhecida produtividade, vasta documentação e ao modelo de "baterias incluídas" que acelera a implementação. Para a persistência dos dados, o \textbf{PostgreSQL} \citep{POSTGRESQL_2025} foi selecionado em função de sua confiabilidade, características de código aberto e, para o desenvolvedor, por uma familiaridade prévia que otimizou a curva de aprendizado inicial. Adicionalmente, foi realizada pesquisa sobre ferramentas de \textbf{conteinerização, com a adoção do Docker} \citep{DOCKER_2025}, para garantir a portabilidade e a consistência do ambiente de desenvolvimento e produção. Conceitos de \textbf{APIs RESTful} também foram estudados para o design da interface de comunicação do back-end, visando a integração com futuras interfaces *front-end*.

    \section{Desenvolvimento e Testes do Sistema Protótipo}
    Esta etapa, que se estendeu de outubro de 2024 a janeiro de 2025, concentrou-se na implementação prática das funcionalidades. Com base nos requisitos levantados junto à Technos, o back-end foi desenvolvido como uma \textbf{aplicação monolítica em Django}. Foram projetados e implementados os modelos de dados e as APIs necessárias para as funcionalidades de autenticação, cadastro de clientes, registro fiscal, controle de demandas/atividades e gerenciamento de estoque. Para garantir a qualidade e estabilidade do código, foram realizados \textbf{testes unitários} em cada um dos modelos e em partes críticas da lógica de negócio. A interface administrativa do Django foi \textbf{customizada} para servir como um painel de controle inicial, permitindo que os membros da empresa júnior interagissem com o sistema desde as fases iniciais e fornecessem *feedback* contínuo.

    \section{Avaliação dos Resultados e Implantação em Ambiente Real}
    De dezembro de 2024 até o presente momento, esta fase envolveu a validação e o uso do sistema em um ambiente real. Inicialmente, o back-end foi implantado na plataforma de nuvem \textbf{Railway} \citep{RAILWAY_2025}, utilizando Docker e \textbf{Gunicorn} \citep{GUNICORN_2025} para gerenciar as requisições, permitindo testes de desempenho e usabilidade em um ambiente acessível. Após o término do plano de testes na Railway, o sistema foi \textbf{migrado para um servidor local} dentro da Universidade Estadual do Maranhão (UEMA), onde foi configurado para operação em rede local. Essa etapa exigiu um aprofundamento em conceitos de \textbf{redes de computadores, orquestração de containers e configuração de servidores}, representando um desafio técnico significativo. A avaliação dos resultados se deu através do \textbf{uso contínuo pelos membros da Technos}, que forneceram *feedback* essencial para identificar necessidades de otimização e planejar futuras melhorias. A partir dos conhecimentos e resultados obtidos, este relatório final e um capítulo de livro foram preparados para documentar e disseminar as descobertas.

\chapter{Resultados}
\label{chap:resultados}
    O sistema desenvolvido adota uma arquitetura monolítica, utilizando o *framework* Django e o sistema gerenciador de banco de dados relacional PostgreSQL. Essa escolha foi motivada pela necessidade de um desenvolvimento ágil e pela possibilidade de reutilizar componentes robustos e seguros já consolidados na estrutura do Django, o que acelerou a implementação das funcionalidades essenciais para a gestão da empresa júnior.

    A interface do sistema foi construída com o Django Admin, customizado para atender às demandas específicas da empresa júnior. Isso permitiu que a equipe tivesse acesso imediato às funcionalidades, eliminando a necessidade de desenvolver uma interface gráfica do zero e reduzindo o tempo de adoção da plataforma.

    \begin{figure}[!htb]
        \centering
        \includegraphics[width=0.8\textwidth]{placeholder_dashboard.png} % Substitua pelo caminho da sua imagem real
        \caption{Visão geral dos aplicativos no dashboard do sistema da Technos.}
        \label{fig:dashboard}
    \end{figure}
    \fonte{O Autor, 2025}

    Para a implementação do sistema, foram desenvolvidos três módulos principais:
    \begin{itemize}[label=\textbullet]
        \item \textbf{Gerenciamento de Clientes:} Responsável pelo armazenamento e gerenciamento de informações dos clientes atendidos pela empresa júnior;
        \item \textbf{Serviços:} Módulo que controla os pedidos realizados, permitindo o acompanhamento do status (pendente, em andamento e concluído);
        \item \textbf{Finanças:} Responsável pelo registro das movimentações financeiras, incluindo entradas e saídas de capital, além da atualização automática do saldo do caixa com base nos pedidos concluídos.
    \end{itemize}

    \begin{figure}[!htb]
        \centering
        \includegraphics[width=0.8\textwidth]{placeholder_pedidos.png} % Substitua pelo caminho da sua imagem real
        \caption{Interface do sistema, demonstrando a tela de Pedidos, de serviços, com detalhes de cada solicitação.}
        \label{fig:pedidos}
    \end{figure}
    \fonte{O Autor, 2025.}

    Cada um desses módulos foi implementado como um aplicativo separado dentro do Django, garantindo organização e modularidade. Para garantir a estabilidade do sistema, foram desenvolvidos testes unitários para cada um dos modelos para possíveis falhas e garantir maior confiabilidade.

    \begin{figure}[!htb]
        \centering
        \includegraphics[width=0.8\textwidth]{placeholder_financas.png} % Substitua pelo caminho da sua imagem real
        \caption{Interface do sistema com informações financeiras e de acesso.}
        \label{fig:financas}
    \end{figure}
    \fonte{O Autor, 2025}

    O sistema foi implantado para testes na plataforma Railway, utilizando um plano gratuito que disponibiliza US\$5,00 em recursos. Para a implantação, foi empregada a conteinerização com Docker, juntamente com a configuração do Gunicorn para o gerenciamento eficiente das requisições. Durante esse processo, foram necessários ajustes nos parâmetros de configuração do Django e do PostgreSQL para garantir o funcionamento adequado em ambiente de produção.

    Nos primeiros testes, o sistema apresentou um tempo médio de resposta de aproximadamente 2,8 segundos, com um consumo máximo de 200 MB de RAM, 0,3 vCPU e 100 MB de armazenamento. Esses valores indicam um funcionamento estável dentro dos limites do plano disponível para o contexto da iniciação científica.

    Atualmente, os membros da empresa júnior já utilizam o sistema para gerenciar clientes, pedidos e movimentações financeiras. A frequência de uso varia conforme a demanda por serviços e a necessidade de atualização dos dados administrativos, demonstrando a aplicabilidade prática da solução desenvolvida.

\chapter{Discussão dos Resultados}
\label{chap:discussao}
    Nesta seção, os resultados obtidos são analisados criticamente, com foco nos desafios enfrentados durante o desenvolvimento do back-end e nas justificativas para as decisões técnicas tomadas, incluindo as perspectivas de futuras otimizações do sistema.

    O sistema desenvolvido, adotando a arquitetura monolítica com Django e PostgreSQL, demonstrou ser eficaz para o propósito inicial de \textbf{agilidade no desenvolvimento e disponibilização de um protótipo funcional}. A escolha do Django proporcionou uma rápida implementação das funcionalidades essenciais, como o gerenciamento de clientes, serviços e finanças, que foram cruciais para o início da organização interna da Technos. Os testes unitários, embora não seguindo uma metodologia TDD (Test-Driven Development) completa, foram fundamentais para garantir a estabilidade e a confiabilidade dos componentes individuais do sistema.

    A \textbf{implantação em produção}, inicialmente na plataforma Railway e, posteriormente, em um servidor local na UEMA, foi um marco importante do projeto, representando o desafio de transpor o ambiente de desenvolvimento para um cenário real de operação. O uso de \textbf{Docker} mostrou-se indispensável para garantir a consistência do ambiente e facilitar o *deploy*. Os dados de performance obtidos na Railway, com tempo médio de resposta de 2,8 segundos e consumo moderado de recursos, validaram a viabilidade do sistema para o volume de uso atual da empresa júnior.

    No entanto, o processo de desenvolvimento e a experiência com o sistema em produção revelaram \textbf{desafios significativos}, especialmente relacionados à arquitetura monolítica e às características da \textit{stack} Python/Django para projetos de maior escala e complexidade:
    \begin{itemize}[label=\textbullet]
        \item \textbf{Acoplamento e Manutenibilidade:} Observou-se que a arquitetura monolítica do Django, embora eficiente para prototipagem rápida, levou a um \textbf{alto acoplamento entre os módulos}. Pequenas alterações em uma parte do código frequentemente impactavam outras seções, tornando a manutenção mais custosa e propensa à introdução de *bugs*. A dificuldade em isolar as funcionalidades comprometia a agilidade em manutenções corretivas e evolutivas, bem como o potencial de escalabilidade modular do sistema a longo prazo.
        \item \textbf{Desafios de Tipagem em Python:} A tipagem dinâmica do Python, que inicialmente contribuiu para a velocidade de desenvolvimento e flexibilidade, mostrou-se um \textbf{obstáculo na depuração de erros em projetos maiores e com maior complexidade lógica}. Problemas relacionados a tipos de dados frequentemente passavam despercebidos em tempo de desenvolvimento e manifestavam-se apenas em tempo de execução, exigindo uma análise demorada de logs e *stack traces*. Em comparação com linguagens de tipagem forte, essa característica, para o contexto do projeto e seu crescimento, aumentou a complexidade na identificação e correção de falhas em um sistema em evolução contínua.
        \item \textbf{Complexidade do Deploy Local e Orquestração:} A migração do ambiente de nuvem para um servidor local na UEMA apresentou desafios consideráveis, exigindo um aprofundamento em \textbf{redes de computadores, configuração de servidores web (ex. Nginx ou Apache como proxy reverso, se aplicável) e orquestração de containers com Docker Compose}. A garantia de que o sistema funcionasse adequadamente em um ambiente interno, lidando com IPs locais, portas e acessibilidade da rede universitária, foi um aprendizado prático intenso na área de DevOps e infraestrutura, fundamental para a operacionalização da solução.
    \end{itemize}

    Esses desafios, que emergiram da experiência prática e do uso contínuo do sistema, levaram à reavaliação da arquitetura e das tecnologias adotadas. A conclusão de que, para garantir a escalabilidade, manutenibilidade e resiliência a longo prazo, uma arquitetura de \textbf{microsserviços com uma \textit{stack} mais robusta e de tipagem forte} seria mais adequada para as futuras demandas da Technos. Essa percepção impulsionou a decisão estratégica de iniciar a \textbf{migração do back-end para Java com o *framework* Spring Boot}, que oferece tipagem forte, padrões de projeto bem estabelecidos, maior controle sobre o ciclo de vida da aplicação e um ecossistema maduro para o desenvolvimento de microsserviços.

    O impacto do sistema na Technos foi imediato e positivo. A \textbf{centralização e padronização} dos processos administrativos, o acesso facilitado às informações de clientes, serviços e finanças, e a \textbf{automação de relatórios} internos, resultaram em uma \textbf{redução significativa de erros manuais} e uma melhoria notável na organização e no controle interno da empresa júnior. Isso permitiu que a equipe da Technos direcionasse seus esforços para suas atividades-fim, otimizando o tempo e os recursos.

    Embora o sistema atual esteja funcionando e entregando valor para a Technos, as otimizações futuras incluirão a implementação de técnicas de *cache* mais avançadas e ajustes nas consultas ao banco de dados para melhorar ainda mais a eficiência e reduzir a latência, principalmente com a transição para a nova arquitetura de microsserviços.

\chapter{Conclusão}
\label{chap:conclusao}
    Este trabalho de Iniciação Científica alcançou seu objetivo geral de \textbf{desenvolver um back-end funcional para um sistema multiplataforma destinado à Technos}, a empresa júnior de Engenharia de Computação da UEMA. O sistema, implementado inicialmente com Python e Django, e utilizando PostgreSQL como sistema gerenciador de banco de dados, demonstrou-se eficaz ao centralizar e padronizar processos essenciais, como o gerenciamento de clientes, serviços e finanças.

    As \textbf{principais contribuições} deste projeto são multifacetadas, evidenciando tanto o desenvolvimento de uma solução tecnológica quanto o aprendizado e a evolução técnica do pesquisador:
    \begin{itemize}[label=\textbullet]
        \item \textbf{Entrega de uma Solução em Produção e Valor para a Empresa:} O back-end encontra-se ativo e em uso pelos membros da Technos desde janeiro de 2025, provendo controle interno aprimorado, facilidade de acesso à informação e automação de relatórios, resultando em um ganho real de eficiência operacional e redução de erros manuais para a empresa júnior.
        \item \textbf{Estabelecimento de uma Base Sólida para o Futuro:} O back-end atual, embora em fase de transição, serve como o alicerce fundamental para o crescimento contínuo da Technos e para futuras expansões do sistema, permitindo a integração de novos módulos e a conexão com diversas interfaces de usuário (*front-ends*).
        \item \textbf{Desenvolvimento de Habilidades Técnicas e Não-Técnicas Aprofundadas:} A experiência prática no desenvolvimento, *deploy* e manutenção do sistema proporcionou um significativo aprendizado em tecnologias de back-end (Django, PostgreSQL, Docker, Gunicorn) e conceitos de DevOps e infraestrutura (*deploy* em nuvem e em ambiente local). Adicionalmente, o contato direto com a Technos para levantamento de requisitos, negociação de funcionalidades e recebimento de *feedback* do usuário aprimorou habilidades interpessoais e de gestão de projetos.
        \item \textbf{Análise Crítica de Arquitetura e Tomada de Decisão Tecnológica:} A identificação dos desafios inerentes à arquitetura monolítica e à tipagem dinâmica de Python, aliada à proatividade na busca por soluções mais adequadas para as demandas de um sistema em crescimento, culminou na decisão estratégica de migrar para uma arquitetura de microsserviços com Java/Spring. Essa transição reflete uma maturidade técnica e a capacidade de adaptar a solução às necessidades de longo prazo do projeto e do mercado.
    \end{itemize}

    Apesar das funcionalidades planejadas para esta fase terem sido integralmente implementadas e o sistema estar em produção, o desenvolvimento revelou \textbf{limitações importantes} na arquitetura monolítica para um sistema que visa escalabilidade e manutenibilidade contínua em um ambiente de empresa. Os desafios de acoplamento entre os módulos e a dificuldade na depuração de erros relacionados à tipagem em Python reforçaram a necessidade de uma abordagem mais robusta e de tipagem forte para projetos de maior complexidade e volume.

    Como \textbf{trabalhos futuros}, as seguintes ações são sugeridas para aprimorar e evoluir o sistema, consolidando os aprendizados e garantindo a sustentabilidade da solução:
    \begin{itemize}[label=\textbullet]
        \item \textbf{Concluir a Migração para Microsserviços com Java/Spring:} Este é o passo mais crítico para garantir a manutenibilidade, escalabilidade e a resiliência do back-end, seguindo as melhores práticas de desenvolvimento e arquitetura de software para sistemas distribuídos.
        \item \textbf{Implementar Estratégias de Testes Abrangentes:} Adotar metodologias como *Test-Driven Development* (TDD) e realizar testes de integração e performance mais rigorosos para garantir a qualidade, a robustez e a conformidade do novo ambiente de microsserviços.
        \item \textbf{Reavaliar o Deploy em Nuvem para Microsserviços:} Uma vez a migração para microsserviços concluída, planejar a implantação em um provedor de nuvem mais escalável e flexível, otimizando o acesso remoto e eliminando as dependências exclusivas da rede local da UEMA.
        \item \textbf{Manutenção do Sistema Legado:} Até a completa transição para a nova \textit{stack}, garantir a manutenção e o suporte ao protótipo em Django que atualmente está em produção, assegurando a continuidade das operações da Technos.
    \end{itemize}

    Em síntese, este projeto de Iniciação Científica não apenas entregou um sistema funcional e valioso para a Technos, mas também proporcionou um ambiente rico de aprendizado e crescimento técnico e profissional, pavimentando o caminho para futuras e mais robustas implementações em Engenharia de Computação.

%---------- Referencias ----------
\clearpage % Necessário para o posicionamento correto da lista de referências
\phantomsection % Garante que o link do sumário para as referências funcione
\addcontentsline{toc}{chapter}{REFERÊNCIAS} % Adiciona "REFERÊNCIAS" ao sumário como capítulo
\printbibliography[heading=none] % Gera a bibliografia sem título de capítulo (pois o ABNT já adiciona)


%---------- Apendices (opcionais) ----------
% Descomente e ajuste conforme necessário
%\apendice
%\chapter{Diagrama de Classes do Sistema Django}
%\label{chap:apendice_diagrama_django}
% [Conteúdo do apêndice, como um diagrama, explicando-o]

% ---------- Anexos (opcionais) ----------
% Descomente e ajuste conforme necessário
%\anexo
%\chapter{Código Fonte do Módulo de Finanças}
%\label{chap:anexo_codigo_financas}
% [Conteúdo do anexo, como trechos de código, explicando-os]

\end{document}
