\documentclass[12pt, a4paper]{article}

% Pacotes essenciais para ABNT e melhorias
\usepackage[utf8]{inputenc} % Codificação de entrada
\usepackage[T1]{fontenc}    % Codificação de fonte
\usepackage{lmodern}        % Fontes mais modernas
\usepackage[brazilian]{babel} % Idioma português do Brasil
\usepackage{indentfirst}    % Indenta o primeiro parágrafo de cada seção
\usepackage{graphicx}       % Para incluir imagens
\usepackage{caption}        % Para legendas de figuras
\usepackage{subcaption}     % Para subfiguras (se necessário)
\usepackage{float}          % Para controlar a posição das figuras (H = here)
\usepackage{amsmath,amssymb} % Para matemática (se precisar de fórmulas)
\usepackage{enumitem}       % Para personalizar listas
\usepackage{geometry}       % Para configurar as margens
\usepackage[hidelinks]{hyperref} % Para links clicáveis no PDF (hidelinks esconde a caixa do link)
\usepackage[backend=biber, style=abnt, natbib=true]{biblatex} % Para bibliografia ABNT com biblatex
\addbibresource{references.bib} % Nome do arquivo .bib que você precisará criar

% Configuração das margens (ABNT)
\geometry{
    a4paper,
    top=3cm,
    bottom=2cm,
    left=3cm,
    right=2cm,
}

% Configuração de espaçamento (ABNT: 1.5 entre linhas)
\linespread{1.5}

% Comando para cabeçalho e rodapé (opcional, dependendo do modelo da UEMA ou do seu orientador)
% \usepackage{fancyhdr}
% \pagestyle{fancy}
% \fancyhf{} % Limpa cabeçalho e rodapé
% \fancyhead[R]{\thepage} % Número da página no canto superior direito
% \renewcommand{\headrulewidth}{0pt} % Remove a linha do cabeçalho

%----------------------------------------------------------------------------------------
% CAPA
%----------------------------------------------------------------------------------------
\begin{document}

\thispagestyle{empty} % Não numera a primeira página

\begin{center}
    \vspace*{\fill}
    \textbf{UNIVERSIDADE ESTADUAL DO MARANHÃO} \\
    \textbf{CENTRO DE CIÊNCIAS TECNOLÓGICAS} \\
    \textbf{DEPARTAMENTO DE ENGENHARIA DE COMPUTAÇÃO}
    \vspace{2cm}

    \textbf{Projeto do Orientador: Desenvolvimento de Um Sistema Multiplataforma para a Empresa Júnior da Engenharia de Computação-UEMA}
    \vspace{1cm}

    \textbf{Plano de Trabalho do Voluntário: Desenvolvimento de um Back-End para um Sistema Multiplataforma para a Empresa Júnior do Curso de Bacharelado em Engenharia de Computação da UEMA}
    \vspace{3cm}

    \textbf{Gabriel Silva Monteiro}
    \vspace{0.5cm}

    \textbf{Prof. Pedro Brandão Neto}
    \vspace{3cm}

    \textbf{São Luís - MA} \\
    \textbf{2025}
    \vspace*{\fill}
\end{center}

\newpage
%----------------------------------------------------------------------------------------
% FOLHA DE ROSTO
%----------------------------------------------------------------------------------------
\begin{center}
    \vspace*{\fill}
    \textbf{Gabriel Silva Monteiro}
    \vspace{2cm}

    \textbf{Desenvolvimento de um Back-End para um Sistema Multiplataforma para a Empresa Júnior do Curso de Bacharelado em Engenharia de Computação da UEMA}
    \vspace{1.5cm}

    Relatório final referente às atividades realizadas conforme o cronograma previsto no plano de trabalho da iniciação científica.

    \vspace{1.5cm}

    Orientador: Prof. Pedro Brandão Neto
    \vspace{3cm}

    \textbf{São Luís - MA} \\
    \textbf{2025}
    \vspace*{\fill}
\end{center}

\newpage
%----------------------------------------------------------------------------------------
% RESUMO
%----------------------------------------------------------------------------------------
\begin{abstract}
    \textbf{RESUMO}

    O presente trabalho aborda o desenvolvimento de um back-end robusto e escalável para um sistema multiplataforma, destinado à Technos, a empresa júnior de Engenharia de Computação da Universidade Estadual do Maranhão (UEMA). O propósito central foi otimizar a gestão interna da empresa, padronizando processos e centralizando informações. A fase inicial do projeto empregou \textbf{Python e o framework Django} para o desenvolvimento de um sistema monolítico, com \textbf{PostgreSQL} como sistema gerenciador de banco de dados relacional. Funcionalidades essenciais foram implementadas, incluindo módulos de autenticação de usuários, gestão de clientes, controle financeiro (registro fiscal e extrato), gerenciamento de demandas/atividades e controle de estoque de materiais. O sistema foi colocado em \textbf{produção em janeiro de 2025}, passando por uma transição de um ambiente de nuvem (Railway) para um \textbf{servidor local na UEMA}, utilizando \textbf{Docker} para conteinerização. Desafios técnicos inerentes à arquitetura monolítica do Django, como o alto acoplamento e dificuldades na depuração de erros decorrentes da tipagem dinâmica do Python, foram identificados e analisados. Tais observações fundamentam a justificativa para uma \textbf{futura migração para uma arquitetura de microsserviços baseada em Java e Spring}, visando aprimorar a manutenibilidade, escalabilidade e performance. Os resultados demonstram que o back-end em operação já contribui significativamente para a organização, acessibilidade de dados e automação de tarefas na Technos, estabelecendo uma base sólida para a evolução contínua da gestão da empresa.

    \vspace{0.5cm}

    \textbf{Palavras-chave}: Empresa júnior, sistema multiplataforma, back-end, Django, PostgreSQL, Docker, Microsserviços, DevOps.
\end{abstract}

\newpage
%----------------------------------------------------------------------------------------
% SUMÁRIO
%----------------------------------------------------------------------------------------
\tableofcontents
\newpage

% Listas de figuras e tabelas
\listoffigures
% \listoftables % Descomente se tiver tabelas
\newpage

% Inicia a numeração das páginas a partir daqui
\pagenumbering{arabic}
\setcounter{page}{5} % Ajusta o número da página inicial, se necessário


%----------------------------------------------------------------------------------------
% 1. INTRODUÇÃO
%----------------------------------------------------------------------------------------
\section{Introdução}
    A Empresa Júnior do curso de Bacharelado em Engenharia de Computação, do Centro de Ciências Tecnológicas (CCT) da Universidade Estadual do Maranhão (UEMA), denominada Technos, tem como objetivo prestar serviço à comunidade interna e externa à universidade, com o treinamento em produtos tecnológicos, conserto e manutenção de equipamentos eletrônicos, com excelência de qualidade e baixo custo, procurando introduzir os participantes na realidade diária de uma empresa prestadora de serviços em hardware e Software.

    A Technos reiniciou suas operações em 2022 com o atendimento aos alunos, professores e laboratório do CCT e de outros centros da UEMA, com os serviços de conserto e manutenção de computadores e equipamentos eletrônicos. Para o desenvolvimento dos trabalhos executados pela Technos, é necessária maior organização e visualização, tanto nas redes sociais quanto nos procedimentos administrativos e técnicos, por meio da construção de um sistema informatizado para registro das atividades, participantes, produtos e para a criação da identidade visual da empresa júnior.

    Uma empresa júnior de Engenharia de Computação de acordo com a Lei 13.267 de 06 de abril de 2016, é uma “entidade organizada nos termos desta Lei, sob a forma de associação civil gerida por estudantes matriculados em cursos de graduação de instituições de ensino superior, com o propósito de realizar projetos e serviços que contribuam para o desenvolvimento acadêmico e profissional dos associados, capacitando-os para o mercado de trabalho” \citep{BRASIL_2016}.

    Para o desenvolvimento dos trabalhos da Technos e a profissionalização das atividades realizadas, é necessário um sistema de gestão de informação e controle dos produtos, serviços, materiais, apresentação da empresa e demais atividades, de forma segura e vinculada às melhores práticas de planejamento e execução do mercado. Assim, um sistema multiplataforma é um passo importante no caminho para o crescimento da empresa júnior e de seus participantes. O back-end dos sistemas computacionais são responsáveis pelo funcionamento do sistema coleta das amostras, com o tratamento das variáveis e a alocação dos dados no banco de dados, além da garantia de integridade dos dados armazenados \citep{SILVA_JUNIOR_2023}.

    A proposta deste trabalho é o desenvolvimento de um back-end para um sistema multiplataforma para a empresa júnior do curso de Bacharelado em Engenharia de Computação da UEMA, a Technos.

%----------------------------------------------------------------------------------------
% 2. OBJETIVOS
%----------------------------------------------------------------------------------------
\section{Objetivos}

\subsection{Objetivos Gerais}
    O objetivo geral do trabalho é desenvolver um back-end para um sistema multiplataforma para a empresa júnior do curso de Bacharelado em Engenharia de Computação da UEMA.

\subsection{Objetivos Específicos}
    \begin{itemize}[label=\textbullet]
        \item Pesquisar sobre as principais tecnologias utilizadas na construção de um back-end para um sistema multiplataforma de acordo com os requisitos da empresa júnior;
        \item Analisar as melhores ferramentas para uso no desenvolvimento do back-end do sistema multiplataforma da empresa júnior;
        \item Implementar as melhores soluções de acordo com as necessidades e requisitos da empresa júnior;
        \item Avaliar o protótipo com a integração do front-end desenvolvido, por meio de teste de software e do uso em um sistema real;
        \item Produzir um capítulo de livro a partir dos trabalhos realizados para submissão e possível publicação.
    \end{itemize}

%----------------------------------------------------------------------------------------
% 3. METODOLOGIA
%----------------------------------------------------------------------------------------
\section{Metodologia}
    A pesquisa e o desenvolvimento do back-end foram conduzidos de forma iterativa, adaptando-se à natureza do projeto de Iniciação Científica e aos requisitos dinâmicos da empresa júnior. O processo foi estruturado em três etapas principais, que permearam o ciclo de vida do projeto:

    \subsection{1ª – Levantamento Bibliográfico e Análise de Tecnologias}
    Esta fase, ocorrida de setembro a novembro de 2024, envolveu uma pesquisa aprofundada sobre as principais tecnologias para o desenvolvimento de back-end em sistemas multiplataforma. O foco inicial recaiu sobre *frameworks* que promovessem agilidade no desenvolvimento e robustez para a aplicação de gestão. A pesquisa levou à escolha de \textbf{Python}, com o *framework* \textbf{Django} \citep{DJANGO_2025}, devido à sua reconhecida produtividade, vasta documentação e ao modelo de "baterias incluídas" que acelera a implementação. Para a persistência dos dados, o \textbf{PostgreSQL} \citep{POSTGRESQL_2025} foi selecionado em função de sua confiabilidade, características de código aberto e, para o desenvolvedor, por uma familiaridade prévia que otimizou a curva de aprendizado inicial. Adicionalmente, foi realizada pesquisa sobre ferramentas de \textbf{conteinerização, com a adoção do Docker} \citep{DOCKER_2025}, para garantir a portabilidade e a consistência do ambiente de desenvolvimento e produção. Conceitos de \textbf{APIs RESTful} também foram estudados para o design da interface de comunicação do back-end, visando a integração com futuras interfaces *front-end*.

    \subsection{2ª – Desenvolvimento e Testes do Sistema Protótipo}
    Esta etapa, que se estendeu de outubro de 2024 a janeiro de 2025, concentrou-se na implementação prática das funcionalidades. Com base nos requisitos levantados junto à Technos, o back-end foi desenvolvido como uma \textbf{aplicação monolítica em Django}. Foram projetados e implementados os modelos de dados e as APIs necessárias para as funcionalidades de autenticação, cadastro de clientes, registro fiscal, controle de demandas/atividades e gerenciamento de estoque. Para garantir a qualidade e estabilidade do código, foram realizados \textbf{testes unitários} em cada um dos modelos e em partes críticas da lógica de negócio. A interface administrativa do Django foi \textbf{customizada} para servir como um painel de controle inicial, permitindo que os membros da empresa júnior interagissem com o sistema desde as fases iniciais e fornecessem *feedback* contínuo.

    \subsection{3ª – Avaliação dos Resultados e Implantação em Ambiente Real}
    De dezembro de 2024 até o presente momento, esta fase envolveu a validação e o uso do sistema em um ambiente real. Inicialmente, o back-end foi implantado na plataforma de nuvem \textbf{Railway} \citep{RAILWAY_2025}, utilizando Docker e \textbf{Gunicorn} \citep{GUNICORN_2025} para gerenciar as requisições, permitindo testes de desempenho e usabilidade em um ambiente acessível. Após o término do plano de testes na Railway, o sistema foi \textbf{migrado para um servidor local} dentro da Universidade Estadual do Maranhão (UEMA), onde foi configurado para operação em rede local. Essa etapa exigiu um aprofundamento em conceitos de \textbf{redes de computadores, orquestração de containers e configuração de servidores}, representando um desafio técnico significativo. A avaliação dos resultados se deu através do \textbf{uso contínuo pelos membros da Technos}, que forneceram *feedback* essencial para identificar necessidades de otimização e planejar futuras melhorias. A partir dos conhecimentos e resultados obtidos, este relatório final e um capítulo de livro foram preparados para documentar e disseminar as descobertas.

%----------------------------------------------------------------------------------------
% 4. RESULTADOS
%----------------------------------------------------------------------------------------
\section{Resultados}
    O sistema desenvolvido adota uma arquitetura monolítica, utilizando o *framework* Django e o sistema gerenciador de banco de dados relacional PostgreSQL. Essa escolha foi motivada pela necessidade de um desenvolvimento ágil e pela possibilidade de reutilizar componentes robustos e seguros já consolidados na estrutura do Django, o que acelerou a implementação das funcionalidades essenciais para a gestão da empresa júnior.

    A interface do sistema foi construída com o Django Admin, customizado para atender às demandas específicas da empresa júnior. Isso permitiu que a equipe tivesse acesso imediato às funcionalidades, eliminando a necessidade de desenvolver uma interface gráfica do zero e reduzindo o tempo de adoção da plataforma.

    \begin{figure}[H]
        \centering
        \includegraphics[width=0.8\textwidth]{placeholder_dashboard.png} % Substitua pelo caminho da sua imagem real
        \caption{Visão geral dos aplicativos no dashboard do sistema da Technos.}
        \label{fig:dashboard}
    \end{figure}
    \textit{Fonte: O Autor, 2025}

    Para a implementação do sistema, foram desenvolvidos três módulos principais:
    \begin{itemize}[label=\textbullet]
        \item \textbf{Gerenciamento de Clientes:} Responsável pelo armazenamento e gerenciamento de informações dos clientes atendidos pela empresa júnior;
        \item \textbf{Serviços:} Módulo que controla os pedidos realizados, permitindo o acompanhamento do status (pendente, em andamento e concluído);
        \item \textbf{Finanças:} Responsável pelo registro das movimentações financeiras, incluindo entradas e saídas de capital, além da atualização automática do saldo do caixa com base nos pedidos concluídos.
    \end{itemize}

    \begin{figure}[H]
        \centering
        \includegraphics[width=0.8\textwidth]{placeholder_pedidos.png} % Substitua pelo caminho da sua imagem real
        \caption{Interface do sistema, demonstrando a tela de Pedidos, de serviços, com detalhes de cada solicitação.}
        \label{fig:pedidos}
    \end{figure}
    \textit{Fonte: O Autor, 2025.}

    Cada um desses módulos foi implementado como um aplicativo separado dentro do Django, garantindo organização e modularidade. Para garantir a estabilidade do sistema, foram desenvolvidos testes unitários para cada um dos modelos para possíveis falhas e garantir maior confiabilidade.

    \begin{figure}[H]
        \centering
        \includegraphics[width=0.8\textwidth]{placeholder_financas.png} % Substitua pelo caminho da sua imagem real
        \caption{Interface do sistema com informações financeiras e de acesso.}
        \label{fig:financas}
    \end{figure}
    \textit{Fonte: O Autor, 2025}

    O sistema foi implantado para testes na plataforma Railway, utilizando um plano gratuito que disponibiliza US\$5,00 em recursos. Para a implantação, foi empregada a conteinerização com Docker, juntamente com a configuração do Gunicorn para o gerenciamento eficiente das requisições. Durante esse processo, foram necessários ajustes nos parâmetros de configuração do Django e do PostgreSQL para garantir o funcionamento adequado em ambiente de produção.

    Nos primeiros testes, o sistema apresentou um tempo médio de resposta de aproximadamente 2,8 segundos, com um consumo máximo de 200 MB de RAM, 0,3 vCPU e 100 MB de armazenamento. Esses valores indicam um funcionamento estável dentro dos limites do plano disponível para o contexto da iniciação científica.

    Atualmente, os membros da empresa júnior já utilizam o sistema para gerenciar clientes, pedidos e movimentações financeiras. A frequência de uso varia conforme a demanda por serviços e a necessidade de atualização dos dados administrativos, demonstrando a aplicabilidade prática da solução desenvolvida.

%----------------------------------------------------------------------------------------
% 5. DISCUSSÃO DOS RESULTADOS
%----------------------------------------------------------------------------------------
\section{Discussão dos Resultados}
    Nesta seção, os resultados obtidos são analisados criticamente, com foco nos desafios enfrentados durante o desenvolvimento do back-end e nas justificativas para as decisões técnicas tomadas, incluindo as perspectivas de futuras otimizações do sistema.

    O sistema desenvolvido, adotando a arquitetura monolítica com Django e PostgreSQL, demonstrou ser eficaz para o propósito inicial de \textbf{agilidade no desenvolvimento e disponibilização de um protótipo funcional}. A escolha do Django proporcionou uma rápida implementação das funcionalidades essenciais, como o gerenciamento de clientes, serviços e finanças, que foram cruciais para o início da organização interna da Technos. Os testes unitários, embora não seguindo uma metodologia TDD (Test-Driven Development) completa, foram fundamentais para garantir a estabilidade e a confiabilidade dos componentes individuais do sistema.

    A \textbf{implantação em produção}, inicialmente na plataforma Railway e, posteriormente, em um servidor local na UEMA, foi um marco importante do projeto, representando o desafio de transpor o ambiente de desenvolvimento para um cenário real de operação. O uso de \textbf{Docker} mostrou-se indispensável para garantir a consistência do ambiente e facilitar o *deploy*. Os dados de performance obtidos na Railway, com tempo médio de resposta de 2,8 segundos e consumo moderado de recursos, validaram a viabilidade do sistema para o volume de uso atual da empresa júnior.

    No entanto, o processo de desenvolvimento e a experiência com o sistema em produção revelaram \textbf{desafios significativos}, especialmente relacionados à arquitetura monolítica e às características da \textit{stack} Python/Django para projetos de maior escala e complexidade:
    \begin{itemize}[label=\textbullet]
        \item \textbf{Acoplamento e Manutenibilidade:} Observou-se que a arquitetura monolítica do Django, embora eficiente para prototipagem rápida, levou a um \textbf{alto acoplamento entre os módulos}. Pequenas alterações em uma parte do código frequentemente impactavam outras seções, tornando a manutenção mais custosa e propensa à introdução de *bugs*. A dificuldade em isolar as funcionalidades comprometia a agilidade em manutenções corretivas e evolutivas, bem como o potencial de escalabilidade modular do sistema a longo prazo.
        \item \textbf{Desafios de Tipagem em Python:} A tipagem dinâmica do Python, que inicialmente contribuiu para a velocidade de desenvolvimento e flexibilidade, mostrou-se um \textbf{obstáculo na depuração de erros em projetos maiores e com maior complexidade lógica}. Problemas relacionados a tipos de dados frequentemente passavam despercebidos em tempo de desenvolvimento e manifestavam-se apenas em tempo de execução, exigindo uma análise demorada de logs e *stack traces*. Em comparação com linguagens de tipagem forte, essa característica, para o contexto do projeto e seu crescimento, aumentou a complexidade na identificação e correção de falhas em um sistema em evolução contínua.
        \item \textbf{Complexidade do Deploy Local e Orquestração:} A migração do ambiente de nuvem para um servidor local na UEMA apresentou desafios consideráveis, exigindo um aprofundamento em \textbf{redes de computadores, configuração de servidores web (ex. Nginx ou Apache como proxy reverso, se aplicável) e orquestração de containers com Docker Compose}. A garantia de que o sistema funcionasse adequadamente em um ambiente interno, lidando com IPs locais, portas e acessibilidade da rede universitária, foi um aprendizado prático intenso na área de DevOps e infraestrutura, fundamental para a operacionalização da solução.
    \end{itemize}

    Esses desafios, que emergiram da experiência prática e do uso contínuo do sistema, levaram à reavaliação da arquitetura e das tecnologias adotadas. A conclusão de que, para garantir a escalabilidade, manutenibilidade e resiliência a longo prazo, uma arquitetura de \textbf{microsserviços com uma \textit{stack} mais robusta e de tipagem forte} seria mais adequada para as futuras demandas da Technos. Essa percepção impulsionou a decisão estratégica de iniciar a \textbf{migração do back-end para Java com o *framework* Spring Boot}, que oferece tipagem forte, padrões de projeto bem estabelecidos, maior controle sobre o ciclo de vida da aplicação e um ecossistema maduro para o desenvolvimento de microsserviços.

    O impacto do sistema na Technos foi imediato e positivo. A \textbf{centralização e padronização} dos processos administrativos, o acesso facilitado às informações de clientes, serviços e finanças, e a \textbf{automação de relatórios} internos, resultaram em uma \textbf{redução significativa de erros manuais} e uma melhoria notável na organização e no controle interno da empresa júnior. Isso permitiu que a equipe da Technos direcionasse seus esforços para suas atividades-fim, otimizando o tempo e os recursos.

    Embora o sistema atual esteja funcionando e entregando valor para a Technos, as otimizações futuras incluirão a implementação de técnicas de *cache* mais avançadas e ajustes nas consultas ao banco de dados para melhorar ainda mais a eficiência e reduzir a latência, principalmente com a transição para a nova arquitetura de microsserviços.

%----------------------------------------------------------------------------------------
% 6. CONCLUSÃO
%----------------------------------------------------------------------------------------
\section{Conclusão}
    Este trabalho de Iniciação Científica alcançou seu objetivo geral de \textbf{desenvolver um back-end funcional para um sistema multiplataforma destinado à Technos}, a empresa júnior de Engenharia de Computação da UEMA. O sistema, implementado inicialmente com Python e Django, e utilizando PostgreSQL como sistema gerenciador de banco de dados, demonstrou-se eficaz ao centralizar e padronizar processos essenciais, como o gerenciamento de clientes, serviços e finanças.

    As \textbf{principais contribuições} deste projeto são multifacetadas, evidenciando tanto o desenvolvimento de uma solução tecnológica quanto o aprendizado e a evolução técnica do pesquisador:
    \begin{itemize}[label=\textbullet]
        \item \textbf{Entrega de uma Solução em Produção e Valor para a Empresa:} O back-end encontra-se ativo e em uso pelos membros da Technos desde janeiro de 2025. Sua implementação proporcionou um controle interno aprimorado, facilidade de acesso à informação e automação de relatórios, resultando em um ganho real de eficiência operacional e redução de erros manuais para a empresa júnior.
        \item \textbf{Estabelecimento de uma Base Sólida para o Futuro:} O back-end atual, embora em fase de transição, serve como o alicerce fundamental para o crescimento contínuo da Technos e para futuras expansões do sistema, permitindo a integração de novos módulos e a conexão com diversas interfaces de usuário (*front-ends*).
        \item \textbf{Desenvolvimento de Habilidades Técnicas e Não-Técnicas Aprofundadas:} A experiência prática no desenvolvimento